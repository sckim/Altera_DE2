\documentclass[psfig,10pt,fullpage]{article}
\addtolength{\textwidth}{1.5in}
\addtolength{\oddsidemargin}{-0.75in}
\addtolength{\topmargin}{-0.75in}
\addtolength{\textheight}{1.5in}
\addtolength{\evensidemargin}{0.75in}
\raggedbottom
\usepackage{psfig,float,times}

\begin{document}

~\\
\centerline{\huge Laboratory Exercise 8}
~\\
\centerline{\large Memory Blocks}
~\\
~\\

In computer systems it is necessary to provide a substantial amount of memory.
If a system is implemented using FPGA technology it is possible
to provide some amount of memory by using the memory resources that exist
in the FPGA device. If additional memory is needed, it has to be implemented
by connecting external memory chips to the FPGA. 
In this exercise we will examine the general issues involved in implementing 
such memory.

A diagram of the random access memory (RAM) module that we will implement is shown in Figure
1{\it a}. It contains 32 eight-bit words (rows), which are accessed using a five-bit
{\it address} port, an eight-bit {\it data} port, and a {\it write} control input. We will
consider two different ways of implementing this memory: using dedicated memory
blocks in an FPGA device, and using a separate memory chip. 

The Cyclone II 2C35 FPGA that
is included on the DE2 board provides dedicated memory resources called {\it M4K
blocks}. Each M4K block contains 4096 memory bits, which can be configured to implement 
memories of various sizes. A common term used to specify the size of a
memory is its {\it aspect ratio}, which gives the {\it depth} in words and the
{\it width} in bits (depth {\sf x} width).
Some aspect ratios supported by the M4K block are 4K {\sf x} 1, 2K {\sf x} 2,
1K {\sf x} 4, and 512 {\sf x} 8. We will utilize the 512 {\sf x} 8 mode in this exercise,
using only the first 32 words in the memory. We should also mention
that many other modes of operation are supported in an M4K block, but we
will not discuss them here.

\begin{figure}[H]
\scriptsize
\centerline{
\hbox{\psfig{figure=figure1.eps}}}
\end{figure}
\centerline{Figure 1.  A 32 {\sf x} 8 RAM module.}
~\\
There are two important features of the M4K block that have to be mentioned.
First, it includes registers that can be used to synchronize all of the 
input and output signals to a clock input. Second, the M4K block has separate
ports for data being written to the memory and data being read from the memory. A
requirement for using the M4K block is that either its input ports, output port, or
both, have to be synchronized to a clock input. Given these requirements,
we will implement the modified 32 {\sf x} 8 RAM module shown in Figure 1{\it b}. It includes 
registers for the {\it address}, {\it data input}, and {\it write} ports, and uses a separate
unregistered {\it data output} port.

~\\
\noindent
{\bf Part I}

~\\
\noindent
Commonly used logic structures, such as adders, registers, counters and memories, 
can be implemented in an FPGA chip by using LPM modules from the Quartus II
Library of Parameterized Modules. Altera recommends that a 
RAM module be implemented by using the {\it altsyncram} LPM. In this exercise you are to
use this LPM to implement the memory module in Figure 1$b$.

\begin{enumerate}
\item Create a new Quartus II project to implement the memory module.
Select as the target chip the Cyclone II EP2C35F672C6, which is the FPGA chip on the 
Altera DE2 board.

\item You can learn how the MegaWizard Plug-in Manager is used to generate a desired 
LPM module by reading the tutorial {\it Using Library Modules in Verilog Designs}. This 
tutorial is provided in the University Program section of Altera's web site.
In the first screen of the MegaWizard Plug-in Manager choose the 
{\it altsyncram} LPM, which is found under the {\sf storage} category. As indicated in Figure 2,
select {\sf Verilog HDL} as the type of output file to create, and give the file the
name {\it ramlpm.v}. On the next page of the Wizard specify a memory size of 32 eight-bit 
words, and select M4K as the type of RAM block. Advance to the subsequent page and
accept the default settings to use a single clock for the RAM's registers, 
and then advance again to 
the page shown in Figure 3. On this page {\it deselect} the setting called {\sf Read
output port(s)} under the category {\sf Which ports should be registered?}. This setting
creates a RAM module that matches the structure in Figure 1$b$, with registered input
ports and unregistered output ports. Accept defaults for the rest of the settings in the
Wizard, and then instantiate in your top-level Verilog file the module generated in 
{\it ramlpm.v}. Include appropriate input and output signals in your Verilog code for the
memory ports given in Figure 1$b$.

\begin{figure}[H]
\scriptsize
\centerline{
\hbox{\psfig{figure=figure2.eps}}}
\end{figure}
\centerline{Figure 2.  Choosing the {\it altsyncram} LPM.}

\begin{figure}[H]
\scriptsize
\centerline{
\hbox{\psfig{figure=figure3.eps}}}
\end{figure}
\centerline{Figure 3.  Configuring input and output ports on the {\it altsyncram} LPM.}
~\\
\item Compile the circuit. Observe in the Compilation Report that the Quartus II 
Compiler uses 256 bits in one of the M4K memory blocks to implement the RAM circuit.
\item Simulate the behavior of your circuit and ensure that you can read and write data in
the memory.
\end{enumerate}

~\\
\noindent
{\bf Part II}

~\\
\noindent
Now, we want to realize the memory circuit in the FPGA on the DE2 board, and 
use toggle switches to load some data into 
the created memory. We also want to display the contents of the RAM on the 7-segment displays.
\begin{enumerate}
\item Make a new Quartus II project which will be used to 
implement the desired circuit on the DE2 board.
\item Create another Verilog file that instantiates the {\it ramlpm} module and that
includes the required input and output pins on the DE2 board. 
Use toggle switches {\it SW}$_{7-0}$ to input a byte of
data into the RAM location identified by a 5-bit address specified with
toggle switches {\it SW}$_{15-11}$. Use {\it SW}$_{17}$ as the {\it Write} signal and
use {\it KEY}$_0$ as the {\it Clock} input. 
Display the value of the {\it Write} signal on {\it LEDG}$_0$. 
Show the address value on the 7-segment displays {\it HEX7} and {\it HEX6}, show the
data being input to the memory on {\it HEX5} and {\it HEX4}, and show the data read out
of the memory on {\it HEX1} and {\it HEX0}. 
\item Test your circuit and make sure that all 32 locations can be loaded properly.
\end{enumerate}

~\\
\noindent
{\bf Part III}

~\\
\noindent
Instead of directly instantiating the LPM module, we can implement the required memory by
specifying its structure in the Verilog code.
In a Verilog-specified design it is possible to define the memory as a
multidimensional array. A 32 {\sf x} 8 array, which has 32 words with
8 bits per word, can be declared by the statement

\begin{center}
reg~[7:0]~~memory\_array~[31:0];
\end{center}

\noindent
In the Cyclone II FPGA, such an array can be implemented either by using
the flip-flops that each logic element contains or, more efficiently, 
by using the M4K blocks.
There are two ways of ensuring that the M4K blocks will be used.
One is to use an LPM module from the Library of Parameterized Modules,
as we saw in Part I.
The other is to define the memory requirement by using a suitable style
of Verilog code from which the Quartus II compiler can infer that a memory
block should be used. Quartus II Help shows how this may be done with examples
of Verilog code (search in the Help for ``Inferred memory''). 

~\\
Perform the following steps:

\begin{enumerate}
\item Create a new project which will be used to implement the desired
circuit on the DE2 board.
\item Write a Verilog file that provides the necessary functionality,
including the ability to load the RAM and read its contents as done in
Part II.
\item Assign the pins on the FPGA to connect to the switches and the 
7-segment displays.
\item Compile the circuit and download it into the FPGA chip.
\item Test the functionality of your design by applying some inputs
and observing the output. Describe any differences you observe in comparison to the
circuit from Part II.
\end{enumerate}

~\\
\noindent
{\bf Part IV}

~\\
\noindent
The DE2 board includes an SRAM chip, called IS61LV25616AL-10, which is a static RAM having
a capacity of 256K 16-bit words. The SRAM interface consists of an 18-bit address
port, A$_{17-0}$, and a 16-bit bidirectional data port, 
I/O$_{15-0}$. It also has several control inputs,
$\overline{CE}$, $\overline{OE}$, $\overline{WE}$, $\overline{UB}$, and $\overline{LB}$, which 
are described in Table~1.
~\\
\begin{center}
\begin{tabular}{c|l}
Name & Purpose \\ \hline
\rule[-0.075in]{0in}{0.25in} $\overline{CE}$ & Chip enable$-$asserted low during all SRAM operations \\ 
$\overline{OE}$ & Output enable$-$can be asserted low during only read operations, or during all operations \\ 
$\overline{WE}$ & Write enable$-$asserted low during a write operation \\ 
$\overline{UB}$ & Upper byte$-$asserted low to read or write the upper byte of an address \\ 
$\overline{LB}$ & Lower byte$-$asserted low to read or write the lower byte of an address \\ 
\end{tabular}
\end{center}

\begin{center}
Table 1. SRAM control inputs.
\end{center}

~\\
\noindent
The operation of the IS61LV25616AL chip is described in its data sheet, which can obtained
from the DE2 System CD that is included with the DE2 board,
or by performing an Internet search. The data sheet describes a
number of modes of operation of the memory and lists many timing parameters related to
its use. For the purposes of this exercise a simple operating mode is to always assert
(set to 0) the control inputs 
$\overline{CE}$, $\overline{OE}$, $\overline{UB}$, and $\overline{LB}$,
and then to control reading and writing of the memory by using only the $\overline{WE}$
input. Simplified timing diagrams that correspond to this mode are given in Figure 4. 
Part ({\it a}) shows a read cycle, which begins when a valid address appears on 
$A_{17-0}$ and the $\overline{WE}$ input is not asserted. The memory places valid 
data on the $I/O_{15-0}$ port after the {\it address access} delay, $t_{AA}$. When 
the read cycle ends because of a change in
the address value, the output data remains valid for the {\it output hold} time, $t_{OHA}$.

\begin{figure}[H]
\scriptsize
\centerline{
\hbox{\psfig{figure=figure8.4.eps}}}
\end{figure}
\centerline{Figure 4.  SRAM read and write cycles.}

~\\
Figure 4$b$ gives the timing for a write cycle. It begins when $\overline{WE}$ is set
to 0, and it ends when $\overline{WE}$ is set back to 1. The address has to be valid for
the {\it address setup} time, $t_{AW}$, and the data to be written has to be
valid for the {\it data setup} time, $t_{SD}$, before the rising edge of $\overline{WE}$.
Table 2 lists the minimum and maximum values of all timing parameters shown in Figure 4.

~\\
\begin{center}
\begin{tabular}{c|c c}
~ & \multicolumn{2}{c}{Value} \\
Parameter & Min & Max\\ \hline
\rule[-0.075in]{0in}{0.25in} $t_{AA}$ & $-$ & 10 ns\\ 
\rule[-0.075in]{0in}{0.25in} $t_{OHA}$ & 3 ns & $-$\\ 
\rule[-0.075in]{0in}{0.25in} $t_{AW}$ & 8 ns & $-$\\ 
\rule[-0.075in]{0in}{0.25in} $t_{SD}$ & 6 ns & $-$\\ 
\rule[-0.075in]{0in}{0.25in} $t_{HA}$ & 0 & $-$\\ 
\rule[-0.075in]{0in}{0.25in} $t_{SA}$ & 0 & $-$\\ 
\rule[-0.075in]{0in}{0.25in} $t_{HD}$ & 0 & $-$\\ 
\end{tabular}
\end{center}

\begin{center}
Table 2. SRAM timing parameter values.
\end{center}

\noindent
You are to realize the 32 {\sf x} 8 memory in Figure 1$a$ by using the SRAM chip. 
It is a good approach to include in your design the registers shown in Figure 1$b$, by
implementing these registers in the FPGA chip.
Be careful to implement properly the bidirectional data port that connects to the memory.

\begin{enumerate}
\item Create a new Quartus II project for your circuit. Write a Verilog file that provides
the necessary functionality, including the ability to load the memory and read its 
contents. Use the same switches, LEDs, and 7-segment displays on the DE2 board 
as in Parts II and III, and use the SRAM pin names shown in Table 3 to interface your 
circuit to the IS61LV25616AL chip (the SRAM pin names are also given in the 
{\it DE2 User Manual}). Note that you will not use all of the address and data ports on the
IS61LV25616AL chip for your 32 {\sf x} 8 memory; connect the unneeded ports to 0 in
your Verilog module.

~\\
\begin{center}
\begin{tabular}{c|l}
SRAM port name & DE2 pin name \\ \hline
\rule[-0.075in]{0in}{0.25in} A$_{17-0}$ & SRAM\_ADDR$_{17-0}$\\ 
I/O$_{15-0}$ & SRAM\_DQ$_{15-0}$\\ 
$\overline{CE}$ & SRAM\_CE\_N\\ 
$\overline{OE}$ & SRAM\_OE\_N\\ 
$\overline{WE}$ & SRAM\_WE\_N\\ 
$\overline{UB}$ & SRAM\_UB\_N\\ 
$\overline{LB}$ & SRAM\_LB\_N\\ 
\end{tabular}
\end{center}

\begin{center}
Table 3. DE2 pin names for the SRAM chip.
\end{center}

\item Compile the circuit and download it into the FPGA chip.
\item Test the functionality of your design by reading and writing values to several
different memory locations.
\end{enumerate}
~\\
\noindent
{\bf Part V}

~\\
The SRAM block in Figure 1 has a single port that provides the address for both read and
write operations. For this part you will create a different type of memory module, in
which there is one port for supplying the address for a read operation, and a separate
port that gives the address for a write operation. Perform the following steps.

\begin{enumerate}
\item Create a new Quartus II project for your circuit. To generate the desired memory
module open the MegaWizard Plug-in Manager and select again the 
{\it altsyncram} LPM in the {\sf storage} category. On Page 1 of the Wizard choose the
setting {\sf With one read port and one write port (simple dual-port mode)} in the category
called {\sf How will you be using the altsyncram?}. Advance through Pages 2 to 5 and make
the same choices as in Part II. On Page 6 choose the setting {\sf I don't care} in the category
{\sf Mixed Port Read-During-Write for Single Input Clock RAM}. This setting specifies that
it does not matter whether the memory outputs the new data being written, or the old data
previously stored, in the case that the write and read addresses are the same.

Page 7 of the Wizard is displayed in Figure 5. It makes use of
a feature that allows the memory module to be loaded with initial data when the circuit is
programmed into the FPGA chip. As shown in the figure, choose the setting {\sf Yes, use this
file for the memory content data}, and specify the filename {\it ramlpm.mif}. To learn
about the format of a {\it memory initialization file} (MIF), see the Quartus II Help.
You will need to create this file and specify some data values to be stored in the memory.
Finish the Wizard and then examine the generated memory module in the file {\it ramlpm.v}.

\begin{figure}[H]
\scriptsize
\centerline{
\hbox{\psfig{figure=figure8.5.eps}}}
\end{figure}
\centerline{Figure 5.  Specifying a memory initialization file (MIF).}

\item Write a Verilog file that instantiates your dual-port memory. 
To see the RAM contents, add to your design a capability to display the
content of each byte (in hexadecimal format) on the 7-segment displays
{\it HEX1} and {\it HEX0}. Scroll through the memory locations by displaying each byte for
about one second. As each byte is being displayed, show its address (in hex format)
on the 7-segment displays {\it HEX3} and {\it HEX2}. Use the 50 MHz clock, {\it CLOCK\_50}, on
the DE2 board, and use {\it KEY}$_0$ as a reset input. For the write address and
corresponding data use the same switches, LEDs, and 7-segment displays as in the previous 
parts of this exercise. Make sure that you properly synchronize the toggle switch inputs
to the 50 MHz clock signal.

\item Test your circuit and verify that the initial contents of the memory match
your {\it ramlpm.mif} file. Make sure that you can independently write data to any
address by using the toggle switches.
\end{enumerate}

~\\
\noindent
{\bf Part VI}

~\\
\noindent
The dual-port memory created in Part V allows simultaneous read and write operations to
occur, because it has two address ports. In this part of the exercise you should create a
similar capability, but using a single-port RAM. Since there will be only one address port
you will need to use multiplexing to select either a read or write address at any specific
time. Perform the following steps.

\begin{enumerate}
\item Create a new Quartus II project for your circuit, and use the
MegaWizard Plug-in Manager to again create a single-port version of the {\it altsyncram}
LPM. For Pages 1 to 6 of the Wizard use the same settings as in Part I. On Page 7, shown
in Figure 6, specify the {\it ramlpm.mif} file as you did in Part V, but also make the
setting {\sf Allow In-System Memory Content Editor to capture and 
update content independently of the
system clock}. This option allows you to use a feature of the Quartus II CAD system called the 
In-System Memory Content Editor to view and manipulate the contents of the created RAM
module. When using this tool you can optionally specify a four-character {\sf `Instance
ID'} that serves as a name for the memory; in Figure 7 we gave the RAM module the
name {\sf 32x8}. Complete the final steps in the Wizard.

\begin{figure}[H]
\scriptsize
\centerline{
\hbox{\psfig{figure=figure8.6.eps}}}
\end{figure}
\centerline{Figure 6.  Configuring {\it altsyncram} for use with the In-System Memory Content Editor.}

\item Write a Verilog file that instantiates your memory module. Include in your design
the ability to scroll through the memory locations as in Part V. Use the same switches,
LEDs, and 7-segment displays as you did previously.

\item Before you can use the In-System Memory Content Editor tool, one additional setting
has to be made. In the 
Quartus II software select {\sf Assignments $>$ Settings} to open the window in Figure
7, and then open the item called
{\sf Default Parameters} under {\sf Analysis and Synthesis
Settings}. As shown in the figure, type the parameter name {\sf CYCLONEII\_SAFE\_WRITE} and
assign the value {\sf RESTRUCTURE}. This parameter allows the Quartus II synthesis tools
to modify the single-port RAM as needed to allow reading and writing of the memory by the 
In-System Memory Content Editor tool. Click {\sf OK} to exit from the Settings window.

\begin{figure}[H]
\scriptsize
\centerline{
\hbox{\psfig{figure=figure8.7.eps}}}
\end{figure}
\centerline{Figure 7.  Setting the {\it CYCLONEII\_SAFE\_WRITE} parameter.}

\item Compile your code and download the circuit onto the DE2 board. Test the circuit's
operation and ensure that read and write operations work properly. Describe any
differences you observe from the behavior of the circuit in Part V.

\item Select {\sf Tools $>$ In-System Memory Content Editor}, which opens the window in
Figure 8. To specify the connection to your DE2 board click on the {\sf Setup} button on the
right side of the screen. In the window in Figure 9 select the 
{\sf USB-Blaster} hardware, and then close the Hardware Setup dialog.

\begin{figure}[H]
\scriptsize
\centerline{
\hbox{\psfig{figure=figure8.8.eps}}}
\end{figure}
\centerline{Figure 8.  The In-System Memory Content Editor window.}

\begin{figure}[H]
\scriptsize
\centerline{
\hbox{\psfig{figure=figure8.9.eps}}}
\end{figure}
\centerline{Figure 9.  The Hardware Setup window.}
~\\

Instructions for using the In-System Memory Content Editor tool can be found in the Quartus II
Help. A simple operation is to right-click on the {\sf 32x8} memory module, as indicated in
Figure 10, and select {\sf Read Data from In-System Memory}. This action causes the
contents of
the memory to be displayed in the bottom part of the window. You can then edit any of the
displayed values by typing over them. To
actually write the new value to the RAM, right click again on the {\sf 32x8} memory
module and select {\sf Write All Modified Words to In-System Memory}.

Experiment by changing some memory values and observing that the data is properly
displayed both on the 7-segment displays on the DE2 board and in the In-System Memory Content 
Editor window.

~\\
\begin{figure}[H]
\scriptsize
\centerline{
\hbox{\psfig{figure=figure8.10.eps}}}
\end{figure}
\centerline{Figure 10.  Using the In-System Memory Content Editor tool.}

\end{enumerate}

~\\
\noindent
{\bf Part VII}

~\\

For this part you are to modify your circuit from Part VI (and Part IV) to use the 
IS61LV25616AL SRAM chip instead of an M4K block. Create a Quartus II project for the new
design, compile it, download it onto the DE2 boards, and test the circuit.

In Part VI you used a memory initialization file to specify the initial contents of 
the 32 {\sf x} 8 RAM block, and you used the In-System Memory Content Editor tool to read and
modify this data. This approach can be used only for the memory resources inside the FPGA
chip. To perform equivalent operations using the external SRAM chip you can use a special
capability of the DE2 board called the {\it DE2 Control Panel}. Chapter 3 of the {\it DE2 User
Manual} shows how to use this tool. The procedure involves programming the FPGA with a
special circuit that communicates with the Control Panel software application, which is
illustrated in Figure 11, and using this setup to load data into the SRAM chip.
Subsequently, you can reprogram the FPGA with your own circuit, which will then 
have access to the data stored in
the SRAM chip (reprogramming the FPGA has no effect on the external memory).
Experiment with this capability and ensure that the results of read and write
operations to the SRAM chip can be observed both in the your circuit and in the DE2
Control Panel software.

~\\
\begin{figure}[H]
\scriptsize
\centerline{
\hbox{\psfig{figure=figure8.11.eps}}}
\end{figure}
\centerline{Figure 11.  The DE2 Control Panel software.}

~\\
~\\
Copyright \copyright 2006 Altera Corporation. 

\end{document}
 

