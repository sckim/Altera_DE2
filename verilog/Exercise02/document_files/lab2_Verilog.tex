\documentclass[psfig,10pt,fullpage]{article}
\addtolength{\textwidth}{1.5in}
\addtolength{\oddsidemargin}{-0.75in}
\addtolength{\topmargin}{-0.75in}
\addtolength{\textheight}{1.5in}
\addtolength{\evensidemargin}{0.75in}
\raggedbottom
\usepackage{psfig,float,times}

\begin{document}

\centerline{\huge Laboratory Exercise 2}
~\\
\centerline{\large Numbers and Displays}
~\\

This is an exercise in designing combinational circuits that can perform
binary-to-decimal number conversion and binary-coded-decimal (BCD) addition.

~\\
\noindent
{\bf Part I}

~\\
\noindent
We wish to display on the 7-segment displays {\it HEX3} to {\it HEX0} the 
values set by the switches $SW_{15-0}$. 
Let the values denoted by $SW_{15-12}$, $SW_{11-8}$, $SW_{7-4}$ and 
$SW_{3-0}$ be displayed on {\it HEX3}, {\it HEX2}, {\it HEX1} and {\it HEX0}, respectively.
Your circuit should be able to display the digits from 0 to 9, and should treat the
valuations 1010 to 1111 as don't-cares.
\begin{enumerate}
\item Create a new project which will be used to implement the desired
circuit on the Altera DE2 board. The intent of this exercise is to manually 
derive the logic functions needed for the 7-segment displays. You should use only
simple Verilog {\bf assign} statements in your code and specify each logic function as
a Boolean expression.
\item Write a Verilog file that provides the necessary functionality. Include this 
file in your project and assign the pins on the FPGA to connect to the 
switches and 7-segment displays, as indicated in the User Manual for the DE2 board.
The procedure for making pin assignments is described in the 
tutorial {\it Quartus II Introduction using Verilog Design}, which is available 
on the {\it DE2 System CD} and in the University Program section of Altera's web site.  
\item Compile the project and download the compiled circuit into the FPGA
chip.
\item Test the functionality of your design by toggling the switches
and observing the displays.
\end{enumerate}

~\\
\noindent
{\bf Part II}

~\\
You are to design a circuit that converts a four-bit binary number $V = v_3 v_2 v_1 v_0$
into its two-digit decimal equivalent $D = d_1 d_0$. Table 1 shows the required output
values. A partial design of this circuit is given in Figure 1. It includes a comparator
that checks when the value of $V$ is greater than 9, and uses the output of this
comparator in the control of the 7-segment displays. You are to complete the design of
this circuit by creating a Verilog module which includes the comparator, multiplexers,
and circuit {\it A} (do not include circuit {\it B} or the 7-segment decoder at this
point). Your Verilog module should have the four-bit input $V$, the four-bit output $M$ and
the output {\it z}. The intent of this exercise is to use simple Verilog {\bf assign}
statements to specify the required logic functions using Boolean
expressions. Your Verilog code should not include any {\bf if-else}, {\bf case}, or
similar statements. 

\begin{center}
\begin{tabular}{c|c c}
Binary value & \multicolumn{2}{c}{Decimal digits}\\ \hline
\hspace{0.75 mm} {\rule[0mm]{0mm}{5mm}0000} & 0 & 0\\ 
\hspace{0.75 mm} 0001 & 0 & 1\\
\hspace{0.75 mm} 0010 & 0 & 2\\
\hspace{0.75 mm} {\rule[0mm]{0mm}{2.5mm}$\ldots$} & $\ldots$ & $\ldots$ \\
\hspace{0.75 mm} {\rule[0mm]{0mm}{5mm}1001} & 0 & 9\\
\hspace{0.75 mm} 1010 & 1 & 0\\
\hspace{0.75 mm} 1011 & 1 & 1\\
\hspace{0.75 mm} 1100 & 1 & 2\\
\hspace{0.75 mm} 1101 & 1 & 3\\
\hspace{0.75 mm} 1110 & 1 & 4\\
\hspace{0.75 mm} 1111 & 1 & 5\\
\end{tabular}
\end{center}

\begin{center}
Table 1. Binary-to-decimal conversion values.
\end{center}

Perform the following steps:

\begin{enumerate}
\item Make a Quartus II project for your Verilog module.
\item Compile the circuit and use functional simulation to verify the correct operation of
your comparator, multiplexers, and circuit {\it A}.
\item Augment your Verilog code to include circuit {\it B} in Figure 1 as well as the 7-segment
decoder. Change the inputs and outputs of your code to use switches $SW_{3-0}$ on the DE2
board to represent the binary number $V$, and the displays {\it HEX1} and {\it HEX0}
to show the values of decimal digits $d_1$ and $d_0$. Make sure to include in your 
project the required pin assignments for the DE2 board.
\item Recompile the project, and then download the circuit into the FPGA chip.
\item Test your circuit by trying all possible values of $V$ and observing the output
displays.
\end{enumerate}

\begin{figure}[H]
\scriptsize
\centerline{
\hbox{\psfig{figure=figure2.1.eps}}}
\end{figure}
~\\
\centerline{Figure 1.  Partial design of the binary-to-decimal conversion circuit.}

~\\
\noindent
{\bf Part III}

~\\
Figure 2$a$ shows a circuit for a {\it full adder}, which has the inputs $a$, $b$, and $c_i$,
and produces the outputs $s$ and $c_o$. Parts $b$ and $c$ of the figure show a circuit
symbol and truth table for the full adder, which produces the two-bit binary sum
$c_o s = a + b + c_i$. Figure 2$d$ shows how four instances of this full adder module
can be used to design a circuit that adds two four-bit numbers. This type of circuit is
usually called a {\it ripple-carry} adder, because of the way that the carry signals are 
passed from one full adder to the next. Write Verilog code that implements this circuit,
as described below.

\begin{figure}[H]
\scriptsize
\centerline{
\hbox{\psfig{figure=figure2.2.eps}}}
\end{figure}
~\\
\centerline{Figure 2.  A ripple-carry adder circuit.}
~\\

\begin{enumerate}
\item Create a new Quartus II project for the adder circuit. Write a Verilog module
for the full adder subcircuit and write a top-level Verilog module that instantiates four 
instances of this full adder.
\item Use switches $SW_{7-4}$ and $SW_{3-0}$ to represent the inputs $A$ and $B$, respectively.
Use $SW_{8}$ for the carry-in $c_{in}$ of the adder. Connect the {\it SW} switches to
their corresponding red lights LEDR, and connect the outputs of the adder, $c_{out}$ and
$S$, to the green lights LEDG.
\item Include the necessary pin assignments for the DE2 board, compile the circuit, and
download it into the FPGA chip.
\item Test your circuit by trying different values for numbers $A$, $B$, and $c_{in}$.
\end{enumerate}

~\\
\noindent
{\bf Part IV}

~\\
In part II we discussed the conversion of binary numbers into decimal digits. It is
sometimes useful to build circuits that use this method of representing decimal numbers,
in which each decimal digit is represented using four bits. This scheme is known as the
{\it binary coded decimal} (BCD) representation. As an example, the decimal value 59 is
encoded in BCD form as 0101 1001.

You are to design a circuit that adds two BCD digits. The inputs to the circuit
are BCD numbers $A$ and $B$, plus a carry-in, $c_{in}$. The output should be a two-digit
BCD sum $S_1 S_0$. Note that the largest sum that needs to be handled by this circuit
is $S_1 S_0 = 9 + 9 + 1 = 19$. Perform the steps given below.

\begin{enumerate}
\item Create a new Quartus II project for your BCD adder. You should use the
four-bit adder circuit from part III to produce a four-bit sum and carry-out for the
operation $A$ + $B$. A circuit that converts this five-bit result, which has the maximum
value 19, into two BCD digits $S_1 S_0$ can be designed in a very
similar way as the binary-to-decimal converter from part II. Write your Verilog code using
simple {\bf assign} statements to specify the required logic functions--do not use 
other types of Verilog
statements such as {\bf if-else} or {\bf case} statements for this part of the exercise.
\item Use switches $SW_{7-4}$ and $SW_{3-0}$ for the inputs $A$ and $B$, respectively, and
use $SW_{8}$ for the carry-in. Connect the {\it SW} switches to
their corresponding red lights LEDR, and connect the four-bit sum and carry-out produced 
by the operation $A$ + $B$ to the green lights LEDG. Display the BCD values of $A$
and $B$ on the 7-segment displays {\it HEX6} and {\it HEX4}, and display the result $S_1 S_0$ on
{\it HEX1} and {\it HEX0}.
\item Since your circuit handles only BCD digits, check for the cases when the input 
$A$ or $B$ is greater than nine. If this occurs, indicate an error by turning on 
the green light {\it LEDG}$_8$.
\item Include the necessary pin assignments for the DE2 board, compile the circuit, and
download it into the FPGA chip.
\item Test your circuit by trying different values for numbers $A$, $B$, and $c_{in}$.
\end{enumerate}

~\\
\noindent
{\bf Part V}

~\\
\noindent
Design a circuit that can add two 2-digit BCD numbers, $A_1 A_0$ and $B_1 B_0$ to produce
the three-digit BCD sum $S_2 S_1 S_0$. Use two instances of your circuit
from part IV to build this two-digit BCD adder. Perform the steps below:
\begin{enumerate}
\item Use switches $SW_{15-8}$ and $SW_{7-0}$ to represent 2-digit 
BCD numbers $A_1 A_0$ and $B_1 B_0$, respectively.
The value of $A_1 A_0$ should be displayed on the 7-segment displays {\it HEX7} 
and {\it HEX6}, while $B_1 B_0$ should be on {\it HEX5} and {\it HEX4}.
Display the BCD sum, $S_2 S_1 S_0$, on the 7-segment displays {\it HEX2},
{\it HEX1} and {\it HEX0}.
\item Make the necessary pin assignments and compile the circuit.
\item Download the circuit into the FPGA chip, and test its operation.
\end{enumerate}

~\\
\noindent
{\bf Part VI}

~\\
In part V you created Verilog code for a two-digit BCD adder by using two instances of the
Verilog code for a one-digit BCD adder from part IV. A different approach for describing
the two-digit BCD adder in Verilog code is to specify an algorithm like the one
represented by the following pseudo-code:

~\\
\begin{center}
\begin{minipage}[t]{12.5 cm}
\begin{tabbing}
ZZZ\=ZZ\=ZZ\=ZZ\=ZZ\=ZZ\=ZZ\=ZZ\=ZZ\=ZZ\=ZZ\kill
1\>$T_0 = A_0 + B_0$ \\
2\>if ($T_0 > 9$) then\\
3\>\>$Z_0 = 10$;\\
4\>\>$c_1 = 1$;\\
5\>else\\
6\>\>$Z_0 = 0$;\\
7\>\>$c_1 = 0$;\\
8\>end if\\
9\>$S_0 = T_0 - Z_0$\\
~\\
10\>$T_1 = A_1 + B_1 + c_1$ \\
11\>if ($T_1 > 9$) then\\
12\>\>$Z_1 = 10$;\\
13\>\>$c_2 = 1$;\\
14\>else\\
15\>\>$Z_1 = 0$;\\
16\>\>$c_2 = 0$;\\
17\>end if\\
18\>$S_1 = T_1 - Z_1$\\
19\>$S_2 = c_2$\\
\end{tabbing}
\end{minipage}
\end{center}
~\\
It is reasonably straightforward to see what circuit could be used to implement this
pseudo-code. Lines 1, 9, 10, and 18 represent adders, lines 2-8 and 11-17 correspond to
multiplexers, and testing for the conditions $T_0 > 9$ and $T_1 > 9$ requires comparators.
You are to write Verilog code that corresponds to this pseudo-code. Note that you can
perform addition operations in your Verilog code instead of the subtractions shown 
in lines 9 and 18. The intent of this part of the exercise is
to examine the effects of relying more on the Verilog compiler to design the circuit by using
{\bf if-else} statements along with the Verilog $>$ and $+$ operators. 
Perform the following steps:

\begin{enumerate}
\item Create a new Quartus II project for your Verilog code. Use the same switches, lights, and
displays as in part~V. Compile your circuit.
\item Use the Quartus II RTL Viewer tool to examine the circuit produced by compiling your
Verilog code. Compare the circuit to the one you designed in Part V.
\item Download your circuit onto the DE2 board and test it by trying different values for 
numbers $A_1 A_0$ and $B_1 B_0$.
\end{enumerate}

~\\
\noindent
{\bf Part VII}

~\\
\noindent
Design a combinational circuit that converts a 6-bit binary number into 
a 2-digit decimal number represented in the BCD form. 
Use switches $SW_{5-0}$ to input the binary number and 7-segment displays 
{\it HEX1} and {\it HEX0} to display the decimal number.
Implement your circuit on the DE2 board and demonstrate its functionality.
~\\
~\\
~\\
~\\
Copyright \copyright 2006 Altera Corporation. 

\end{document}
