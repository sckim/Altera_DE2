\documentclass[psfig,10pt,fullpage]{article}
\addtolength{\textwidth}{1.5in}
\addtolength{\oddsidemargin}{-0.75in}
\addtolength{\topmargin}{-0.75in}
\addtolength{\textheight}{1.5in}
\addtolength{\evensidemargin}{0.75in}
\raggedbottom
\usepackage{psfig,float,times}

\begin{document}

~\\
~\\
~\\
\centerline{\huge Laboratory Exercise 5}
~\\
\centerline{\large Clocks and Timers}
~\\
~\\

This is an exercise in implementing and using a real-time clock.

~\\
\noindent
{\bf Part I}

~\\
\noindent
Implement a 3-digit BCD counter. Display the contents of the counter 
on the 7-segment displays, {\it HEX2$-$0}. Derive a control signal,
from the 50-MHz clock signal provided on the Altera DE2 board, 
to increment the contents of the counter at one-second intervals. 
Use the pushbutton switch {\it KEY}$_0$ to reset the counter to 0.
\begin{enumerate}
\item Create a new Quartus II project which will be used to implement the desired
circuit on the DE2 board.
\item Write a Verilog file that specifies the desired circuit.
\item Include the Verilog file in your project and compile the circuit.
\item Simulate the designed circuit to verify its functionality.
\item Assign the pins on the FPGA to connect to the 7-segment
displays and the pushbutton switch, 
as indicated in the User Manual for the DE2 board.
\item Recompile the circuit and download it into the FPGA chip.
\item Verify that your circuit works correctly by observing the display.
\end{enumerate}

~\\
\noindent
{\bf Part II}

~\\
\noindent
Design and implement a circuit on the DE2 board that acts as a time-of-day clock.
It should display the hour (from 0 to 23) on the 7-segment displays 
{\it HEX7$-$6}, the minute (from 0 to 60) on {\it HEX5$-$4} and the second (from 0 to 60)
on {\it HEX3$-$2}. Use the switches {\it SW}$_{15-0}$ to preset the hour and minute 
parts of the time displayed by the clock.


~\\
\noindent
{\bf Part III}

~\\
\noindent
Design and implement on the DE2 board a reaction-timer circuit. The circuit is to operate
as follows:
\begin{enumerate}
\item The circuit is reset by pressing the pushbutton switch {\it KEY}$_0$.
\item After an elapsed time, the red light labeled {\it LEDR}$_0$ turns on and
a four-digit BCD counter starts counting in intervals of milliseconds. 
The amount of time in seconds from when the circuit is reset until {\it LEDR}$_0$
is turned on is set by switches {\it SW}$_{7-0}$.
\item A person whose reflexes are being
tested must press the pushbutton {\it KEY}$_3$ as quickly as possible to turn the
LED off and freeze the counter in its present state. The count which
shows the reaction time will be displayed on the 7-segment displays {\it HEX2-0}.
\end{enumerate}
~\\
~\\
~\\
~\\
~\\
~\\
Copyright \copyright 2006 Altera Corporation. 

\end{document}
