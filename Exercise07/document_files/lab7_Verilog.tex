\documentclass[epsfig,10pt,fullpage]{article}
\addtolength{\textwidth}{1.5in}
\addtolength{\oddsidemargin}{-0.75in}
\addtolength{\topmargin}{-0.75in}
\addtolength{\textheight}{1.5in}
\addtolength{\evensidemargin}{0.75in}
\raggedbottom
\usepackage{ae,aecompl}
\usepackage{epsfig,float,times}

\begin{document}

~\\
~\\
~\\
\centerline{\huge Laboratory Exercise 7}
~\\
\centerline{\large Finite State Machines}
~\\
~\\
\noindent
This is an exercise in using finite state machines.

~\\
\noindent
{\bf Part I}
~\\
~\\
\noindent
We wish to implement a finite state machine (FSM) that recognizes two specific sequences 
of applied input symbols, namely four consecutive 1s or four consecutive 0s. There is an 
input {\it w} and an output {\it z}. Whenever $w = 1$
or $w = 0$ for four consecutive clock pulses the value of {\it z} has to be 1; 
otherwise, $z = 0$. Overlapping sequences
are allowed, so that if $w = 1$ for five consecutive clock pulses the output {\it z}
will be equal to 1 after the fourth and
fifth pulses. Figure 1 illustrates the required relationship between {\it w} and {\it z}.

~\\
\begin{figure}[H]
\scriptsize
\centerline{
\hbox{\psfig{figure=figure7.1.eps}}}
\end{figure}
\centerline{Figure 1.  Required timing for the output {\it z}.}
~\\
~\\
A state diagram for this FSM is shown in Figure 2. For this part you are to manually
derive an FSM circuit that implements this state diagram, including the logic expressions
that feed each of the state flip-flops. To
implement the FSM use nine state flip-flops called $y_8, \ldots, y_0$ and the 
one-hot state assignment given in Table 1.

~\\
\begin{center}
\begin{tabular}{c|c}
~ & State Code \\ 
Name & $y_8 y_7 y_6 y_5 y_4 y_3 y_2 y_1 y_0$ \\ \hline
\rule[-0.075in]{0in}{0.25in}{\bf A} & $000000001$ \\ 
{\bf B} & $000000010$ \\ 
{\bf C} & $000000100$ \\ 
{\bf D} & $000001000$ \\ 
{\bf E} & $000010000$ \\ 
{\bf F} & $000100000$ \\ 
{\bf G} & $001000000$ \\ 
{\bf H} & $010000000$ \\ 
{\bf I} & $100000000$ \\ 
\end{tabular}
\end{center}

\begin{center}
Table 1. One-hot codes for the FSM.
\end{center}

\begin{figure}[H]
\scriptsize
\centerline{
\hbox{\psfig{figure=figure7.2.eps}}}
\end{figure}
\centerline{Figure 2.  A state diagram for the FSM.}
~\\
~\\
Design and implement your circuit on the DE2 board as follows.

\begin{enumerate}
\item Create a new Quartus II project for the FSM circuit. Select as the target chip the 
Cyclone II EP2C35F672C6, which is the FPGA chip on the Altera DE2 board.

\item Write a Verilog file that instantiates the nine flip-flops in the circuit and which
specifies the logic expressions that drive the flip-flop input ports. Use only
simple {\bf assign} statements in your Verilog code to specify the logic feeding the
flip-flops. Note that the one-hot code enables you to derive these expressions by
inspection.

Use the toggle switch {\it SW}$_0$ on the Altera DE2 board as an active-low synchronous reset input
for the FSM, use {\it SW}$_1$ as the {\it w} input, and the pushbutton {\it KEY}$_0$ as the clock input which 
is applied manually.  Use the green LED {\it LEDG}$_0$ as the output {\it z}, 
and assign the state flip-flop outputs to the red LEDs {\it LEDR}$_8$ to {\it LEDR}$_0$.

\item Include the Verilog file in your project, and assign the pins on the FPGA to 
connect to the switches and the LEDs, as indicated in the User Manual for the DE2 board.
Compile the circuit.

\item Simulate the behavior of your circuit.

\item Once you are confident that the circuit works properly as a result of your
simulation, download the circuit into the FPGA chip.  Test the functionality of your 
design by applying the input sequences and observing the output LEDs. Make sure that the
FSM properly transitions between states as displayed on the red LEDs, and that it produces
the correct output values on {\it LEDG}$_0$.

\item Finally, consider
a modification of the one-hot code given in Table 1. When an FSM is going to be
implemented in an FPGA, the circuit can often be simplified if all flip-flop
outputs are 0 when the FSM is in the reset state. This approach is preferable because the
FPGA's flip-flops usually include a {\it clear} input port, which can be conveniently
used to realize the reset state, but the flip-flops often do not include a {\it set} input port.

\newpage
Table 2 shows a modified one-hot state assignment in which the reset state, {\it A},
uses all 0s. This is accomplished by inverting the state variable $y_0$. 
Create a modified version of your Verilog code that implements this state
assignment. ({\it Hint}: you should need to make very few changes to the logic expressions
in your circuit to implement the modified codes.) Compile your new circuit and test it both
through simulation and by downloading it onto the DE2 board.

\begin{center}
\begin{tabular}{c|c}
~ & State Code \\ 
Name & $y_8 y_7 y_6 y_5 y_4 y_3 y_2 y_1 y_0$ \\ \hline
\rule[-0.075in]{0in}{0.25in}{\bf A} & $000000000$ \\ 
{\bf B} & $000000011$ \\ 
{\bf C} & $000000101$ \\ 
{\bf D} & $000001001$ \\ 
{\bf E} & $000010001$ \\ 
{\bf F} & $000100001$ \\ 
{\bf G} & $001000001$ \\ 
{\bf H} & $010000001$ \\ 
{\bf I} & $100000001$ \\ 
\end{tabular}
\end{center}

\begin{center}
Table 2. Modified one-hot codes for the FSM.
\end{center}

\end{enumerate}

\noindent
{\bf Part II}

~\\
\noindent
For this part you are to write another style of Verilog code for the FSM in Figure 2. In
this version of the code you should not manually derive the logic expressions needed for
each state flip-flop. Instead, describe the state table for the FSM by using a
Verilog {\bf case} statement in an {\bf always} block, and use another {\bf always} block to
instantiate the state flip-flops. You can use a third {\bf always} block or simple assignment
statements to specify the output {\it z}. To implement the FSM, use four state flip-flops
$y_3, \ldots, y_0$ and binary codes, as shown in Table 3.

\begin{center}
\begin{tabular}{c|c}
~ & State Code \\ 
Name & $y_3 y_2 y_1 y_0$ \\ \hline
\rule[-0.075in]{0in}{0.25in}{\bf A} & $0000$ \\ 
{\bf B} & $0001$ \\ 
{\bf C} & $0010$ \\ 
{\bf D} & $0011$ \\ 
{\bf E} & $0100$ \\ 
{\bf F} & $0101$ \\ 
{\bf G} & $0110$ \\ 
{\bf H} & $0111$ \\ 
{\bf I} & $1000$ \\ 
\end{tabular}
\end{center}

\begin{center}
Table 3. Binary codes for the FSM.
\end{center}

\noindent
A suggested skeleton of the Verilog code is given in Figure 3.

\begin{center}
\begin{minipage}[t]{12.5 cm}
\begin{tabbing}
ZZ\=ZZ\=ZZ\=ZZ\=ZZ\=ZZ\=ZZ\=ZZ\=ZZ\=ZZ\=ZZ\kill
{\bf module} ~part2 (~$\ldots$~);\\
\>$\ldots$ define input and output ports\\
~\\
\>$\ldots$ define signals\\
\>{\bf reg} [3:0] y\_Q, Y\_D;  \>\>\>\>\>\>\>\>// y\_Q represents current state, Y\_D represents next state\\
\>{\bf parameter} A = 4'b0000, B = 4'b0001, C = 4'b0010, D = 4'b0011, E = 4'b0100,\\
\>\>F = 4'b0101, G = 4'b0110, H = 4'b0111, I = 4'b1000;\\
~\\
\>{\bf always} @(w, y\_Q)\\
\>{\bf begin}: state\_table\\
\>\>{\bf case} (y\_Q)\\
\>\>\>A:	\>{\bf if} (!w) Y\_D = B;\\
\>\>\>\>{\bf else} Y\_D = F;\\
\>\>\>$\ldots$ remainder of state table \\
\>\>\>{\bf default}: Y\_D = 4'bxxxx;\\
\>\>{\bf endcase}\\
\>{\bf end} // state\_table\\
\\
\>{\bf always} @({\bf posedge} Clock)\\
\>{\bf begin}: state\_FFs\\
\>\>$\ldots$ \\
\>{\bf end} // state\_FFS\\
\\
\>$\ldots$ assignments for output z and the LEDs\\
{\bf endmodule}
~\rule{5.0in}{0in}\\
\end{tabbing}
\end{minipage}
\end{center}

\begin{center}
Figure 3. Skeleton Verilog code for the FSM.
\end{center}
~\\
Implement your circuit as follows.

\begin{enumerate}
\item Create a new project for the FSM. Select as the target chip the 
Cyclone II EP2C35F672C6.

\item Include in the project your Verilog file that uses the style of code in Figure 3.
Use the toggle switch {\it SW}$_0$ on the Altera DE2 board as an active-low synchronous reset input
for the FSM, use {\it SW}$_1$ as the {\it w} input, and the pushbutton {\it KEY}$_0$ as the clock input which 
is applied manually.  Use the green LED {\it LEDG}$_0$ as the output {\it z}, 
and assign the state flip-flop outputs to the red LEDs {\it LEDR}$_3$ to {\it LEDR}$_0$.
Assign the pins on the FPGA to 
connect to the switches and the LEDs, as indicated in the User Manual for the DE2 board.

\item Before compiling your code it is necessary to explicitly tell the Synthesis tool in
Quartus II that you wish to have the finite state machine implemented using the state
assignment specified in your Verilog code. If you do not explicitly give this
setting to Quartus II, the Synthesis tool will automatically use a state assignment of
its own choosing, and it will ignore the state codes specified in your Verilog code. To
make this setting, choose {\sf Assignments > Settings} in Quartus II, and then click on the
{\sf Analysis and Synthesis} item on the left side of the window. As indicated in Figure 4,
change the parameter {\sf State Machine Processing} to the setting {\sf User-Encoded}.

\item To examine the circuit produced by Quartus II open the RTL Viewer tool. Double-click
on the box shown in the circuit that represents the finite state machine, and determine
whether the state diagram that it shows properly corresponds to the one in Figure 2.
To see the state codes used for your FSM, open the Compilation Report, select the {\sf Analysis
and Synthesis} section of the report, and click on {\sf State Machines}.

\item Simulate the behavior of your circuit.

\item Once you are confident that the circuit works properly as a result of your
simulation, download the circuit into the FPGA chip.  Test the functionality of your 
design by applying the input sequences and observing the output LEDs. Make sure that the
FSM properly transitions between states as displayed on the red LEDs, and that it produces
the correct output values on {\it LEDG}$_0$.

\item In step 3 you instructed the Quartus II Synthesis tool to use the state
assignment given in your Verilog code. To see the result of removing this setting, open
again the Quartus II settings window by choosing {\sf Assignments > Settings}, and 
click on the {\sf Analysis and Synthesis} item. Change the setting for  
{\sf State Machine Processing} from {\sf User-Encoded} to {\sf One-Hot}. Recompile the
circuit and then open the report file, select the {\sf Analysis
and Synthesis} section of the report, and click on {\sf State Machines}.
Compare the state codes shown to those given in Table 2, and
discuss any differences that you observe.
\end{enumerate}

\begin{figure}[H]
\scriptsize
\centerline{
\hbox{\psfig{figure=figure7.4.eps}}}
\end{figure}
\centerline{Figure 4.  Specifying the state assignment method in Quartus II.}
~\\
\noindent
{\bf Part III}
~\\
~\\
\noindent
For this part you are to implement the sequence-detector FSM by using shift registers,
instead of using the more formal approach described above. Create Verilog code that
instantiates two 4-bit shift registers; one is for recognizing a sequence of four 0s, and
the other for four 1s. Include the appropriate logic expressions in your design
to produce the output {\it z}. Make a Quartus II project for your design and implement
the circuit on the DE2 board. Use the switches and LEDs on the board in a similar way as
you did for Parts I and II and observe the behavior of your shift registers and the
output {\it z}. Answer the following question: could you use just one 4-bit shift
register, rather than two? Explain your answer.

\newpage
\noindent
{\bf Part IV}

~\\
\noindent
We want to design a modulo-10 counter-like circuit that behaves as follows.
It is reset to 0 by the {\it Reset} input.
It has two inputs, $w_1$ and $w_0$, which control its counting operation.
If $w_1 w_0 = 00$, the count remains the same. If $w_1 w_0 = 01$, the count
is incremented by 1. If $w_1 w_0 = 10$, the count is incremented by 2.
If $w_1 w_0 = 11$, the count is decremented by 1. All changes take place on
the active edge of a {\it Clock} input.
Use toggle switches {\it SW}$_2$ and {\it SW}$_1$ for inputs $w_1$ and $w_0$.
Use toggle switch {\it SW}$_0$ as an active-low synchronous reset, and use
the pushbutton {\it KEY}$_0$ as a manual clock.
Display the decimal contents of the counter on the 7-segment display {\it HEX0}.
\begin{enumerate}
\item Create a new project which will be used to implement the 
circuit on the DE2 board.
\item Write a Verilog file that defines the circuit. Use the style of code indicated
in Figure 3 for your FSM.
\item Include the Verilog file in your project and compile the circuit.
\item Simulate the behavior of your circuit.
\item Assign the pins on the FPGA to connect to the switches and the 7-segment
display.
\item Recompile the circuit and download it into the FPGA chip.
\item Test the functionality of your design by applying some inputs
and observing the output display.
\end{enumerate}

~\\
\noindent
{\bf Part V}

~\\
\noindent
For this part you are to design a circuit for the DE2 board that
scrolls the word "HELLO" in ticker-tape fashion
on the eight 7-segment displays {\it HEX}$7-0$. The 
letters should move from right to left each time you apply a manual clock pulse 
to the circuit. After the word "HELLO" scrolls off the left side of the displays 
it then starts again on the right side. 

Design your circuit by using eight 7-bit registers connected in a queue-like fashion,
such that the outputs of the first register feed the inputs of the
second, the second feeds the third, and so on. This type of connection between registers
is often called a {\it pipeline}. Each register's outputs
should directly drive the seven segments of one display. You are to design a finite state
machine that controls the pipeline in two ways:

\begin{enumerate}
\item
For the first eight clock pulses after the system is reset, the FSM 
inserts the correct characters (H,E,L,L,0,~,~,~) into the first of the 7-bit registers
in the pipeline.
\item
After step 1 is complete, the FSM configures the pipeline into
a loop that connects the last register back to the first one, 
so that the letters continue to scroll indefinitely.
\end{enumerate}

\noindent
Write Verilog code for the ticker-tape circuit and create a Quartus II
project for your design. Use {\it KEY}$_0$
on the DE2 board to clock the FSM and pipeline registers and use 
{\it SW}$_0$ as a synchronous active-low reset input. Write Verilog code 
in the style shown in Figure 3 for your finite state machine.

Compile your Verilog code, download it onto the DE2 board and test the circuit. 

~\\
\noindent
{\bf Part VI}

~\\
\noindent
For this part you are to modify your circuit from Part V so that it no longer
requires manually-applied clock pulses. 
Your circuit should scroll the word "HELLO" such that the
letters move from right to left in intervals of about one second.
Scrolling should continue indefinitely; after the word "HELLO" scrolls
off the left side of the displays it should start again on the right side. 

\noindent
Write Verilog code for the ticker-tape circuit and create a Quartus II
project for your design. Use the 50-MHz clock signal, {\it CLOCK\_50},
on the DE2 board to clock the FSM and pipeline registers and use 
{\it KEY}$_0$ as a synchronous active-low reset input. Write Verilog code 
in the style shown in Figure 3 for your finite state machine, 
and ensure that all flip-flops in your circuit are clocked directly by the 
{\it CLOCK\_50} input. Do not derive or use any other clock signals 
in your circuit.

Compile your Verilog code, download it onto the DE2 board and test the circuit. 

~\\
\noindent
{\bf Part VII}

~\\
\noindent
Augment your design from Part VI so that under the control of pushbuttons
{\it KEY}$_2$ and {\it KEY}$_1$ the rate at which the letters move from right to left
can be changed. If {\it KEY}$_1$ is pressed, the letters should move twice as fast.
If {\it KEY}$_2$ is pressed, the rate has to be reduced by a factor of 2.

Note that the {\it KEY}$_2$ and {\it KEY}$_1$ switches are debounced and will produce exactly one
low pulse when pressed. However, there is no way of knowing how long a switch may
remain depressed, which means that the pulse duration can be arbitrarily long. A good
approach for designing this circuit is to include a second FSM in your Verilog 
code that properly responds to the pressed keys. The outputs of this FSM can change
appropriately when a key is pressed, and the FSM can wait for each key press to end
before continuing. The outputs produced by this second FSM can be used as
part of the scheme for creating a variable time interval in your circuit. Note
that {\it KEY}$_2$ and {\it KEY}$_1$ are asynchronous inputs to your circuit, so
be sure to synchronize them to the clock signal before using these signals as
inputs to your finite state machine.

The ticker tape should operate as follows. When the circuit is reset, scrolling 
occurs at about one second intervals. Pressing {\it KEY}$_1$ repeatedly causes the 
scrolling speed to double to a maximum of four letters per second.
Pressing {\it KEY}$_2$ repeatedly causes the 
scrolling speed to slow down to a minimum of one letter every four seconds.

Implement your circuit on the DE2 board and demonstrate that it works properly.

\vskip 2in
Copyright \copyright 2006 Altera Corporation. 

\end{document}
