\documentclass[psfig,10pt,fullpage]{article}
\addtolength{\textwidth}{1.5in}
\addtolength{\oddsidemargin}{-0.75in}
\addtolength{\topmargin}{-0.75in}
\addtolength{\textheight}{1.5in}
\addtolength{\evensidemargin}{0.75in}
\raggedbottom
\usepackage{psfig,float,times}

\begin{document}

~\\
~\\
~\\
\centerline{\huge Laboratory Exercise 6}
~\\
\centerline{\large Adders, Subtractors, and Multipliers}
~\\
~\\

The purpose of this exercise is to examine arithmetic circuits that add, subtract,
and multiply numbers. Each type of circuit will be implemented in two ways: first by writing
Verilog code that describes the required functionality, and second by making use
of predefined subcircuits from Altera's library of parameterized modules (LPMs). The
results produced for various implementations will be compared, both in terms of the circuit
structure and its speed of operation.

~\\
\noindent
{\bf Part I}
~\\

\noindent
Consider again the four-bit ripple-carry adder circuit that was used in lab exercise 2;
a diagram of this circuit is reproduced in Figure 1$a$. You are to create an 8-bit 
version of the adder and include it in the circuit shown in Figure~1$b$. Your circuit
should be designed to support signed numbers in 2's-complement form, and 
the {\it Overflow} output should be set to 1 whenever the sum
produced by the adder does not provide the correct signed value. Perform the 
steps shown below.

\begin{figure}[H]
\scriptsize
\centerline{
\hbox{\psfig{figure=figure6.1.eps}}}
\end{figure}
~\\
\centerline{Figure 1.  An 8-bit signed adder with registered inputs and outputs.}
~\\
\begin{enumerate}
\item Make a new Quartus II project and write Verilog code that describes the circuit in
Figure 1$b$. Use the circuit structure in Figure 1$a$ to describe your adder.
\item Include the required input and output ports in your project to implement the adder
circuit on the DE2 board. Connect the inputs $A$ and $B$ to switches {\it SW}$_{15-8}$ and 
{\it SW}$_{7-0}$, respectively. Use {\it KEY}$_0$ as an active-low asynchronous reset
input, and use {\it KEY}$_1$ as a manual clock input. Display the sum outputs of the adder
on the red {\it LEDR}$_{7-0}$ lights and display the overflow output on the green 
{\it LEDG}$_8$ light. The hexadecimal values of $A$ and $B$ should be shown on the 
displays {\it HEX7-6} and {\it HEX5-4}, and the hexadecimal value of $S$ should appear on
{\it HEX1-0}.
\item Compile your code and use timing simulation to verify the correct operation of
the circuit. Once the simulation works properly, download the circuit onto the DE2 board
and test it by using different values of $A$ and $B$. Be sure to check for proper
functionality of the {\it Overflow} output.
\item Open the Quartus II Compilation Report and examine the results reported by the
Timing Analyzer. What is the maximum operating frequency, {\it fmax}, of your circuit? What is
the longest path in the circuit in terms of delay?
\end{enumerate}

~\\
\noindent
{\bf Part II}
~\\

\noindent
Modify your circuit from Part I so that it can perform both addition and subtraction of
eight-bit numbers. Use switch {\it SW}$_{16}$ to specify whether addition or subtraction
should be performed. Connect the other switches, lights, and displays as described for
Part I.

\begin{enumerate}
\item Simulate your adder/subtractor circuit to show that it functions properly, and then
download it onto the DE2 board and test it by using different switch settings.
\item Open the Quartus II Compilation Report and examine the results reported by the
Timing Analyzer. What is the fmax of your circuit? What is
the longest path in the circuit in terms of delay?
\end{enumerate}

~\\
\noindent
{\bf Part III}
~\\

\noindent
Repeat Part I using the predefined adder circuit called {\it lpm\_add\_sub}, instead of your
ripple-carry adder structure from Figure 1. The {\it lpm\_add\_sub} module can be found in
Altera's library of parameterized modules (LPMs), which is provided as part of the Quartus
II system.  The procedure for using these predefined modules in Quartus II projects 
is described in the tutorial {\it Using Library Modules in Verilog Designs}, which is available 
on the {\it DE2 System CD} and in the University Program section of Altera's web site.  
\begin{enumerate}
\item Configure the {\it lpm\_add\_sub} module so that it performs only addition, to make
the functionality comparable to Part I. Store your configuration of the {\it lpm\_add\_sub} 
module in the file {\it lpm\_add8.v}.  After instantiating this module in your 
Verilog code, compile the project and use the Quartus II Chip Editor tool to examine some 
of the details of the implemented circuit. 

One way to examine the adder subcircuit using the Chip Editor tool is illustrated in
Figure 2. In the Quartus II Project Navigator window right-click on the part of your
circuit hierarchy that represents the {\it lpm\_add8} subcircuit, and select the command
{\sf Locate} $>$ {\sf Locate in Chip Editor}. This opens the Chip Editor window 
shown in Figure 3. The logic elements in the Cyclone II FPGA that are used to implement
the adder are highlighted in blue in the Chip Editor tool.
Position your mouse pointer over any of these logic elements and double-click 
to open the Resource Property Editor window displayed in Figure 4. In the box labeled 
{\sf Node name} you can select any of the nine logic elements that implement the adder 
module. The Resource Property Editor allows you to examine the contents of a logic element and
to see how one logic element is connected others.

\begin{figure}[H]
\scriptsize
\centerline{
\hbox{\psfig{figure=figure6.2.eps}}}
\end{figure}
~\\
\centerline{Figure 2.  Locating the eight-bit adder in the Chip Editor tool.}
~\\

\begin{figure}[H]
\scriptsize
\centerline{
\hbox{\psfig{figure=figure6.3.eps}}}
\end{figure}
~\\
\centerline{Figure 3.  The highlighted logic elements for the eight-bit adder.}
~\\

Using the tools described above, and referencing the Data Sheet information for the
Cyclone II FPGA, describe the eight-bit adder circuit implemented with
the {\it lpm\_add\_sub} module.

~\\
\begin{figure}[H]
\scriptsize
\centerline{
\hbox{\psfig{figure=figure6.4.eps}}}
\end{figure}
~\\
\centerline{Figure 4.  Examining details in a logic element using the Resource Property
Editor.}
~\\

\item Open the Quartus II Compilation Report and and compare the fmax of your adder 
circuit with the one designed in Part I. Discuss any differences in performance that are observed.
\end{enumerate}

~\\
\noindent
{\bf Part IV}
~\\

\noindent
Repeat Part II using the predefined adder circuit called {\it lpm\_add\_sub}, instead of your
adder-subtractor circuit based on Figure 1.

Comment briefly on the circuit structure obtained using the LPM module, and compare 
the fmax of this circuit to the one from Part II. Describe how the 
{\it lpm\_add\_sub} module implements the {\it Overflow} signal.

\newpage
\noindent
~\\
~\\
~\\
{\bf Part V}
~\\

\noindent
Figure 5$a$ gives an example of the traditional paper-and-pencil multiplication $P = A
\times B$, where $A = 12$ and $B = 11$. We need to add two summands that are shifted versions of
$A$ to form the product $P = 132$. Part $b$ of the figure shows the same example using
four-bit binary numbers. Since each digit in $B$ is either 1 or 0, the summands are either
shifted versions of $A$ or 0000. Figure 5$c$ shows how each summand can be formed by using
the Boolean AND operation of $A$ with the appropriate bit in $B$.
~\\

\begin{figure}[H]
\scriptsize
\centerline{
\hbox{\psfig{figure=figure6.5.eps}}}
\end{figure}
~\\
\centerline{Figure 5.  Multiplication of binary numbers.}
~\\

~\\
A four-bit circuit that implements $P = A \times B$ is illustrated in Figure 6. Because of its
regular structure, this type of multiplier circuit is usually called an {\it array
multiplier}. The shaded areas in the circuit correspond to the shaded columns in Figure
5$c$. In each row of the multiplier AND gates are used to produce the summands,
and full adder modules are used to generate the required sums.

~\\
\begin{figure}[H]
\scriptsize
\centerline{
\hbox{\psfig{figure=figure6.6.eps}}}
\end{figure}
~\\
\centerline{Figure 6.  An array multiplier circuit.}
~\\

Use the following steps to implement the array multiplier circuit:

\begin{enumerate}
\item Create a new Quartus II project which will be used to implement the desired
circuit on the Altera DE2 board.
\item Generate the required Verilog file, include it in your project, 
and compile the circuit.
\item Use functional simulation to verify that your code is correct.
\item Augment your design to use switches $SW_{11-8}$ to represent the 
number $A$ and switches $SW_{3-0}$ to represent $B$. The hexadecimal values of $A$ 
and $B$ are to be displayed on the 7-segment displays {\it HEX6} and {\it HEX4}, respectively.
The result ~$P = A \times B$~ is to be displayed on {\it HEX1} and {\it HEX0}.
\item Assign the pins on the FPGA to connect to the switches and 7-segment displays,
as indicated in the User Manual for the DE2 board.
\item Recompile the circuit and download it into the FPGA chip.
\item Test the functionality of your design by toggling the switches
and observing the 7-segment displays.
\end{enumerate}

\pagebreak
\noindent
{\bf Part VI}

~\\
\noindent
Extend your multiplier to multiply 8-bit numbers and produce a 16-bit product. 
Use switches $SW_{15-8}$ to represent the number $A$ and switches $SW_{7-0}$ to 
represent $B$. The hexadecimal values of $A$ and $B$ are to be displayed on the 
7-segment displays {\it HEX7$-$6} and {\it HEX5$-$4}, respectively.
The result ~$P = A \times B$~ is to be displayed on {\it HEX3$-$0}.
Add registers to your circuit to store the values of $A$, $B$, and the product $P$, using
a similar structure as shown for the registered adder in Figure 1.

After successfully compiling and testing your multiplier circuit, examine the results
produced by the Quartus II Timing Analyzer to determine the fmax of your circuit. 
What is the longest path in terms of delay between registers?

~\\
\noindent
{\bf Part VII}

~\\
\noindent
Change your Verilog code to implement the 8 {\sf x} 8 multiplier by using the
{\it lpm\_mult} module from the library of parameterized modules in the Quartus II
system. Complete the design steps above. Compare the results in terms of the 
number of logic elements (LEs) needed and the circuit fmax.

~\\
\noindent
{\bf Part VIII}

~\\
\noindent
It many applications of digital circuits it is useful to be able to perform some number of
multiplications and then produce a summation of the results. For this part of the exercise
you are to design a circuit that performs the calculation
$$
S = (A \times B) + (C \times D)
$$
\noindent
The inputs $A$, $B$, $C$, and $D$ are eight-bit unsigned numbers, and $S$ provides a
16-bit result. Your circuit should also provide a carry-out signal, $C_{out}$. All of the
inputs and outputs of the circuit should be registered, similar to the structure
shown in Figure 1$b$.

\begin{enumerate}
\item Create a new Quartus II project which will be used to implement the desired
circuit on the Altera DE2 board. Use the {\it lpm\_mult} and {\it lpm\_add\_sub} modules
to realize the multipliers and adders in your design.
\item Connect the inputs $A$ and $C$ to switches {\it SW}$_{15-8}$ and 
connect the inputs $B$ and $D$ to switches {\it SW}$_{7-0}$. Use switch 
{\it SW}$_{16}$ to select between these two sets of inputs: $A$, $B$ or $C$, $D$. Also, use
the switch {\it SW}$_{17}$ as a {\it write enable} (WE) input. Setting {\it WE} to 1 should 
allow data to be loaded into the input registers when an active clock edge occurs, while 
setting {\it WE} to 0 should prevent loading of these registers.
\item Use {\it KEY}$_0$ as an active-low asynchronous reset
input, and use {\it KEY}$_1$ as a manual clock input. 
\item
Display the hexadecimal value of either $A$ or $C$, as selected by {\it SW}$_{16}$, 
on displays {\it HEX7-6} and display either $B$ or $D$ on {\it HEX5-4}. The sum $S$ should
be shown on {\it HEX3-0}, and the $C_{out}$ signal should appear on {\it LEDG}$_8$. 
\item Compile your code and use either functional or timing simulation to verify that 
your circuit works properly. Then download the circuit onto the DE2 board and test its
operation.
\item It is often necessary to ensure that a digital circuit is able to meet certain
speed requirements, such as a particular frequency of a signal applied to a clock input.
Such requirements
are provided to a CAD system in the form of {\it timing constraints}. The procedure for
using timing constraints in the Quartus II CAD system is described in the tutorial
{\it Timing Considerations with Verilog-Based Designs}, which is available 
on the {\it DE2 System CD} and in the University Program section of Altera's web site.  

For this exercise we are using a manual clock that is applied by a pushbutton switch,
so no realistic timing requirements exist. But to demonstrate the design issues
involved, assume that your circuit is required to operate with a clock frequency of 220
MHz. Enter this frequency as a timing constraint in the Quartus II software, and recompile
your project. The Timing Analyzer should report that it is unable to meet the timing
requirements due to the lengths of various register-to-register paths in the circuit. 
Examine the timing
analysis report and describe briefly the timing violations observed.
\item
One way to increase the speed of operation of a given circuit is to insert registers into the
circuit in a way that shortens the lengths of its longest paths. This technique is 
referred to as {\it pipelining} a circuit, and the inserted registers are often called
{\it pipeline registers}. Insert pipeline registers into your design between the
multipliers and the adder. Recompile your project and discuss the results obtained.
\end{enumerate}

~\\
\noindent
{\bf Part IX}

~\\
\noindent
The Quartus II software includes a predesigned module called {\it altmult\_add} that can
perform calculations of the form $S = (A \times B) + (C \times D)$. Repeat Part VIII using
this module instead of the {\it lpm\_mult} and {\it lpm\_add\_sub} modules. Test your
circuit using both simulation and by downloading the circuit onto the DE2 board.

Briefly describe how the implementation of your circuit differs when using the
{\it altmult\_add} module. Examine its performance both with and without the pipeline 
registers discussed in Part VIII.

~\\
~\\
~\\
~\\
~\\
~\\
~\\
~\\
Copyright \copyright 2006 Altera Corporation. 

\end{document}
