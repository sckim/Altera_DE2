\documentclass[psfig,10pt,fullpage]{article}
\addtolength{\textwidth}{1.5in}
\addtolength{\oddsidemargin}{-0.75in}
\addtolength{\topmargin}{-0.75in}
\addtolength{\textheight}{1.5in}
\addtolength{\evensidemargin}{0.75in}
\raggedbottom
\usepackage{psfig,float,times}

\begin{document}

~\\
~\\
~\\
\centerline{\huge Laboratory Exercise 1}
~\\
\centerline{\large Switches, Lights, and Multiplexers}
~\\
~\\

The purpose of this exercise is to learn how to connect simple input and
output devices to an FPGA chip and implement a circuit that uses these devices.
We will use the switches $SW_{17-0}$ on the DE2 board as inputs to the circuit.
We will use light emitting diodes (LEDs) and 7-segment displays as output
devices.

~\\
\noindent
{\bf Part I}
~\\
~\\
\noindent
The DE2 board provides 18 toggle switches, called {\it SW}$_{17-0}$, that can be used as
inputs to a circuit, and 18 red lights, called {\it LEDR}$_{17-0}$, that can be used to
display output values. Figure 1 shows a simple VHDL entity that uses these switches 
and shows their states on the LEDs. Since there are 18 switches and lights
it is convenient to represent them as arrays in the VHDL code, as shown. We have used a
single assignment statement for all 18 {\it LEDR} outputs, which is equivalent to the 
individual assignments

\begin{center}
\begin{minipage}[t]{12.5 cm}
\begin{tabbing}
LEDR(17) $<$= SW(17);\\
LEDR(16) $<$= SW(16);\\
$\ldots$\\
LEDR(0) $<$= SW(0);
\end{tabbing}
\end{minipage}
\end{center}
The DE2 board has hardwired connections between its FPGA chip and the switches and
lights. To use {\it SW}$_{17-0}$ and {\it LEDR}$_{17-0}$ it is necessary to include in
your Quartus II project the correct pin assignments, which are given in the {\it DE2 
User Manual}. For example, the manual specifies that {\it SW}$_0$ is connected to the
FPGA pin {\it N25} and {\it LEDR}$_0$ is connected to pin {\it AE23}. A good way to
make the required pin assignments is to import into the Quartus II software the file called
{\it DE2\_pin\_assignments.csv}, which is provided on the {\it DE2 System CD} and 
in the University Program section of
Altera's web site.  The procedure for making pin assignments is described in the 
tutorial {\it Quartus II Introduction using VHDL Design}, which is also available from
Altera.

It is important to realize that the pin assignments in the {\it DE2\_pin\_assignments.csv}
file are useful only if the pin names given in the file are exactly the same as the port names
used in your VHDL entity. The file uses the names {\it SW}[0] $\ldots$ {\it SW}[17] 
and {\it LEDR}[0] $\ldots$ {\it LEDR}[17] for the switches and lights, which is the reason
we used these names in Figure 1 (note that the Quartus II software 
uses [ ] square brackets for array elements, while
the VHDL syntax uses ( ) round brackets).

~\\
\begin{center}
\begin{minipage}[t]{12.5 cm}
\begin{tabbing}
LIBRARY ieee;\\
USE ieee.std\_logic\_1164.all;\\
~\\
-\,- Simple module that connects the SW switches to the LEDR lights\\
ZZ\=PORT (~\=SWZZ \=: OUTZZ\=STD\_LOGIC;\kill
ENTITY part1 IS \\
\>PORT (\>SW \>: IN \>STD\_LOGIC\_VECTOR(17 DOWNTO 0);\\
\>\>LEDR \>: OUT \>STD\_LOGIC\_VECTOR(17 DOWNTO 0));  -\,- red LEDs\\
END part1;\\
~\\
ZZ\=ZZ\=ZZ\=ZZ\=ZZ\=ZZ\=ZZ\=ZZ\=ZZ\=ZZ\=ZZ\kill
ARCHITECTURE Behavior OF part1 IS\\
BEGIN\\
\>LEDR $<$= SW;\\
END Behavior\\
\end{tabbing}
\end{minipage}
\end{center}

\begin{center}
Figure 1. VHDL code that uses the DE2 board switches and lights.
\end{center}

~\\
Perform the following steps to implement a circuit corresponding to the code
in Figure 1 on the DE2 board.
\begin{enumerate}
\item Create a new Quartus II project for your circuit. Select Cyclone II EP2C35F672C6
as the target chip, which is the FPGA chip on the Altera DE2 board. 
\item Create a VHDL entity for the code in Figure 1 and include it in your project.
\item Include in your project the required pin assignments for the DE2 board, as discussed
above. Compile the project.
\item Download the compiled circuit into the FPGA chip. Test the functionality of the 
circuit by toggling the switches and observing the LEDs.
\end{enumerate}

~\\
\noindent
{\bf Part II}
~\\
~\\
\noindent
Figure 2$a$ shows a sum-of-products circuit that implements
a 2-to-1 {\it multiplexer} with a select input $s$.
If $s = 0$ the multiplexer's output $m$ is equal to the input $x$, and if $s=1$ the
output is equal to $y$. Part $b$ of the figure gives a truth table for this
multiplexer, and part $c$ shows its circuit symbol. 

\begin{figure}[H]
\scriptsize
\centerline{
\hbox{\psfig{figure=figure1.2.eps}}}
\end{figure}
~\\
\centerline{Figure 2.  A 2-to-1 multiplexer.}
~\\
~\\

The multiplexer can be described by the following VHDL statement:

\begin{center}
\begin{minipage}[t]{12.5 cm}
\begin{tabbing}
m $<$= (NOT (s) AND x) OR (s AND y);
\end{tabbing}
\end{minipage}
\end{center}

You are to write a VHDL entity that includes eight
assignment statements like the one shown above to describe the circuit given in Figure
3$a$. This circuit has two eight-bit inputs, $X$ and $Y$, and produces the eight-bit output
$M$. If $s=0$ then $M = X$, while if $s=1$ then $M=Y$. We refer to this circuit as an eight-bit
wide 2-to-1 multiplexer. It has the circuit symbol shown in Figure 3$b$, in which $X$,
$Y$, and $M$ are depicted as eight-bit wires. Perform the steps shown below.

\begin{figure}[H]
\scriptsize
\centerline{
\hbox{\psfig{figure=figure1.3.eps}}}
\end{figure}
~\\
\centerline{Figure 3.  An eight-bit wide 2-to-1 multiplexer.}
~\\

\begin{enumerate}
\item Create a new Quartus II project for your circuit.
\item Include your VHDL file for the eight-bit wide 2-to-1 multiplexer 
in your project. Use switch {\it SW}$_{17}$ on the DE2 board as the $s$ input, switches
{\it SW}$_{7-0}$ as the $X$ input and 
{\it SW}$_{15-8}$ as the $Y$ input. Connect the {\it SW} switches
to the red lights {\it LEDR} and connect the output $M$ to the green lights {\it LEDG}$_{7-0}$.
\item Include in your project the required pin assignments for the DE2 board. As discussed
in Part I, these 
assignments ensure that the input ports of your VHDL code will use the pins on the Cyclone 
II FPGA that are connected to the {\it SW} switches, and the output ports of your VHDL code
will use the FPGA pins connected to the {\it LEDR} and {\it LEDG} lights. 
\item Compile the project.
\item Download the compiled circuit into the FPGA chip. Test the functionality of the 
eight-bit wide 2-to-1 multiplexer by toggling the switches and observing the LEDs.
\end{enumerate}

~\\
\noindent
{\bf Part III}

~\\
\noindent
In Figure 2 we showed a 2-to-1 multiplexer that selects between the two inputs {\it x} and {\it
y}. For this part consider a circuit in which the output {\it m} has to be selected from
five inputs $u$, $v$, $w$, $x$, and $y$. Part $a$ of Figure 4 shows how we can build the
required 5-to-1 multiplexer by using four 2-to-1 multiplexers. The circuit uses a 3-bit
select input $s_2 s_1 s_0$ and implements the truth table shown in Figure 4$b$. A circuit
symbol for this multiplexer is given in part $c$ of the figure. 

Recall from Figure 3
that an eight-bit wide 2-to-1 multiplexer can be built by using eight instances of a 2-to-1
multiplexer. Figure 5 applies this concept to define a three-bit wide 5-to-1 multiplexer. It
contains three instances of the circuit in Figure 4$a$.

\begin{figure}[H]
\scriptsize
\centerline{
\hbox{\psfig{figure=figure1.4.eps}}}
\end{figure}
~\\
\centerline{Figure 4.  A 5-to-1 multiplexer.}
~\\

\begin{figure}[H]
\scriptsize
\centerline{
\hbox{\psfig{figure=figure1.5.eps}}}
\end{figure}
~\\
\centerline{Figure 5.  A three-bit wide 5-to-1 multiplexer.}
~\\

Perform the following steps to implement the three-bit wide 5-to-1 multiplexer.
\begin{enumerate}
\item Create a new Quartus II project for your circuit.
\item Create a VHDL entity for the three-bit wide 5-to-1 multiplexer. Connect its select
inputs to switches {\it SW}$_{17-15}$, and use the remaining 15 switches SW$_{14-0}$ to
provide the five 3-bit inputs $U$ to $Y$. Connect the {\it SW} switches
to the red lights {\it LEDR} and connect the output $M$ to the green lights {\it LEDG}$_{2-0}$.
\item Include in your project the required pin assignments for the DE2 board.
Compile the project.
\item Download the compiled circuit into the FPGA chip. Test the functionality of the 
three-bit wide 5-to-1 multiplexer by toggling the switches and observing the LEDs. Ensure
that each of the inputs $U$ to $Y$ can be properly selected as the output $M$.
\end{enumerate}

~\\
\noindent
{\bf Part IV}

~\\
\noindent
Figure 6 shows a {\it 7-segment} decoder module that has the three-bit input $c_2 c_1
c_0$. This decoder produces seven outputs that are used to display a character on a
7-segment display. Table 1 lists the characters that should be displayed for each
valuation of $c_2 c_1 c_0$. To keep the design simple, only four characters
are included in the table (plus the `blank' character, which is selected for codes
$100-111$).

The seven segments in the display are identified by the indices 0 to 6 shown in the
figure. Each segment is illuminated by driving it to the logic value 0. You are to write a
VHDL entity that implements logic functions that represent circuits needed to activate
each of the seven segments. Use only
simple VHDL assignment statements in your code to specify each logic function using a
Boolean expression. 

~\\
\begin{figure}[H]
\scriptsize
\centerline{
\hbox{\psfig{figure=figure1.6.eps}}}
\end{figure}
~\\
\centerline{Figure 6.  A 7-segment decoder.}
~\\

~\\
\begin{center}
\begin{tabular}{l|c}
$c_2 c_1 c_0$ & Character \\ \hline
\hspace{0.75 mm} {\rule[0mm]{0mm}{5mm}000} & H\\ 
\hspace{0.75 mm} 001 & E\\
\hspace{0.75 mm} 010 & L\\
\hspace{0.75 mm} 011 & O\\
\hspace{0.75 mm} 100 & \\
\hspace{0.75 mm} 101 & \\
\hspace{0.75 mm} 110 & \\
\hspace{0.75 mm} 111 & \\
\end{tabular}
\end{center}

\begin{center}
Table 1. Character codes.
\end{center}

~\\
Perform the following steps:

\begin{enumerate}
\item Create a new Quartus II project for your circuit.
\item Create a VHDL entity for the 7-segment decoder. Connect the $c_2 c_1 c_0$ inputs
to switches {\it SW}$_{2-0}$, and connect the outputs of the decoder to the {\it HEX0} 
display on the DE2 board. The segments in this display are called 
{\it HEX0}$_0$, {\it HEX0}$_1$, $\ldots$, {\it HEX0}$_6$, corresponding to Figure 6.
You should declare the 7-bit port 

\begin{center}
\begin{minipage}[t]{12.5 cm}
\begin{tabbing}
HEX0 : OUT STD\_LOGIC\_VECTOR(0 TO 6);
\end{tabbing}
\end{minipage}
\end{center}

in your VHDL code so that the
names of these outputs match the corresponding names in the {\it DE2 User Manual} and the 
{\it DE2\_pin\_assignments.csv} file.
\item After making the required DE2 board pin assignments, compile the project.
\item Download the compiled circuit into the FPGA chip. Test the functionality of the 
circuit by toggling the {\it SW}$_{2-0}$ switches and observing the 7-segment display.
\end{enumerate}

~\\
\noindent
{\bf Part V}

~\\
\noindent
Consider the circuit shown in Figure 7. It uses a three-bit wide 5-to-1 multiplexer to
enable the selection of five characters that are displayed on a 7-segment display. Using the
7-segment decoder from Part IV this circuit can display any of the characters H, E, L, O,
and `blank'. The character codes are set according to Table 1 by using the switches 
{\it SW}$_{14-0}$, and a specific character is selected for display by setting the
switches {\it SW}$_{17-15}$.

An outline of the VHDL code that represents this circuit is 
provided in Figure 8. Note that we have
used the circuits from Parts III and IV as subcircuits in this code. You are to
extend the code in Figure 8 so that it uses five 7-segment displays rather than just one.
You will need to use five instances of each of the subcircuits. The
purpose of your circuit is to display any word on the five displays that is composed of the
characters in Table 1, and be able to rotate this word in a circular fashion across the
displays when the switches {\it SW}$_{17-15}$ are toggled. As an example,
if the displayed word is HELLO, then your circuit should produce the output patterns
illustrated in Table 2.

~\\
\begin{figure}[H]
\scriptsize
\centerline{
\hbox{\psfig{figure=figure1.7.eps}}}
\end{figure}
~\\
\centerline{Figure 7.  A circuit that can select and display one of five characters.}
~\\

~\\
\begin{center}
\begin{minipage}[t]{12.5 cm}
\begin{tabbing}
LIBRARY ieee;\\
USE ieee.std\_logic\_1164.all;\\
~\\
ZZ\=PORT (~\=SWZZ \=: OUTZZ\=STD\_LOGIC;\kill
ENTITY part5 IS \\
\>PORT (\>SW \>: IN \>STD\_LOGIC\_VECTOR(17 DOWNTO 0);\\
\>\>HEX0	\>: OUT \>STD\_LOGIC\_VECTOR(0 TO 6));\\
END part5;\\
~\\
ARCHITECTURE Behavior OF part5 IS\\
ZZ\=ZZ\=PORT (~\=S, U, V, W, X, YZZ \=: OUTZZ\=STD\_LOGIC;\kill
\>COMPONENT mux\_3bit\_5to1\\
\>\>PORT (\>S, U, V, W, X, Y	\>: IN \>STD\_LOGIC\_VECTOR(2 DOWNTO 0);\\
\>\>\>M \>: OUT \>STD\_LOGIC\_VECTOR(2 DOWNTO 0));\\
\>END COMPONENT;\\
ZZ\=ZZ\=PORT (~\=DisplayZZ \=: OUTZZ\=STD\_LOGIC;\kill
\>COMPONENT char\_7seg\\
\>\>PORT (\>C \>: IN	\>STD\_LOGIC\_VECTOR(2 DOWNTO 0);\\
\>\>\>Display \>: OUT \>STD\_LOGIC\_VECTOR(0 TO 6));\\
\>END COMPONENT;\\
ZZ\=ZZ\=ZZ\=ZZ\=ZZ\=ZZ\=ZZ\=ZZ\=ZZ\=ZZ\=ZZ\kill
\>SIGNAL M : STD\_LOGIC\_VECTOR(2 DOWNTO 0);\\
BEGIN\\
\>M0: mux\_3bit\_5to1 PORT MAP (SW(17 DOWNTO 15), SW(14 DOWNTO 12), SW(11 DOWNTO 9),\\
\>\>SW(8 DOWNTO 6), SW(5 DOWNTO 3), SW(2 DOWNTO 0), M);\\
\>H0: char\_7seg PORT MAP (M, HEX0);\\
END Behavior;\\
~\\
LIBRARY ieee;\\
USE ieee.std\_logic\_1164.all;\\
~\\
-\,- implements a 3-bit wide 5-to-1 multiplexer\\
ZZ\=PORT (~\=S, U, V, W, X, YZZ \=: OUTZZ\=STD\_LOGIC;\kill
ENTITY mux\_3bit\_5to1 IS\\
\>PORT (\>S, U, V, W, X, Y	\>: IN \>STD\_LOGIC\_VECTOR(2 DOWNTO 0);\\
\>\>M	\>: OUT \>STD\_LOGIC\_VECTOR(2 DOWNTO 0));\\
END mux\_3bit\_5to1;\\
~\\
ARCHITECTURE Behavior OF mux\_3bit\_5to1 IS\\
~\\
\>$\ldots$ code not shown\\
~\\
END Behavior;\\
~\\
LIBRARY ieee;\\
USE ieee.std\_logic\_1164.all;\\
~\\
ZZ\=PORT (~\=DisplayZZ \=: OUTZZ\=STD\_LOGIC;\kill
ENTITY char\_7seg IS\\
\>PORT (\>C \>: IN \>STD\_LOGIC\_VECTOR(2 DOWNTO 0);\\
\>\>Display	\>: OUT \>STD\_LOGIC\_VECTOR(0 TO 6));\\
END char\_7seg;\\
~\\
ARCHITECTURE Behavior OF char\_7seg IS\\
~\\
\>$\ldots$ code not shown\\
~\\
END Behavior;
\end{tabbing}
\end{minipage}
\end{center}

\begin{center}
Figure 8. VHDL code for the circuit in Figure 7.
\end{center}

~\\
\begin{center}
\begin{tabular}{l|ccccc}
{\it SW}$_{17}$ {\it SW}$_{16}$ {\it SW}$_{15}$ & \multicolumn{5}{c}{Character pattern} \\
\hline
\hspace{8.0 mm} {\rule[0mm]{0mm}{5mm}000} & H & E & L & L & O\\ 
\hspace{8.0 mm} 001 & E & L & L & O & H\\
\hspace{8.0 mm} 010 & L & L & O & H & E\\
\hspace{8.0 mm} 011 & L & O & H & E & L\\
\hspace{8.0 mm} 100 & O & H & E & L & L\\
\end{tabular}
\end{center}

\begin{center}
Table 2. Rotating the word HELLO on five displays.
\end{center}
~\\

Perform the following steps.
\begin{enumerate}
\item Create a new Quartus II project for your circuit.
\item Include your VHDL entity in the Quartus II project. Connect the switches 
{\it SW}$_{17-15}$ to the select inputs of each of the five instances of the three-bit 
wide 5-to-1 multiplexers. Also connect {\it SW}$_{14-0}$ to each instance of the
multiplexers as required to produce the patterns of characters shown in Table 2.
Connect the outputs of the five multiplexers to the 7-segment displays {\it HEX4}, 
{\it HEX3}, {\it HEX2}, {\it HEX1}, and {\it HEX0}.
\item Include the required pin assignments for the DE2 board for all switches, LEDs, 
and 7-segment displays. Compile the project.
\item Download the compiled circuit into the FPGA chip. Test the functionality of the 
circuit by setting the proper character codes on the switches SW$_{14-0}$ and then 
toggling {\it SW}$_{17-15}$ to observe the rotation of the characters.
\end{enumerate}

~\\
\noindent
{\bf Part VI}

~\\
\noindent
Extend your design from Part V so that is uses all eight 7-segment displays on the DE2
board. Your circuit should be able to display words with five (or fewer) characters 
on the eight displays, and rotate the displayed word when the switches {\it SW}$_{17-15}$ 
are toggled. If the displayed word is HELLO, then your circuit should produce the patterns 
shown in Table 3.

~\\
\begin{center}
\begin{tabular}{l|cccccccc}
{\it SW}$_{17}$ {\it SW}$_{16}$ {\it SW}$_{15}$ & \multicolumn{8}{c}{Character pattern} \\
\hline
\hspace{8.0 mm} {\rule[0mm]{0mm}{5mm}000} &  &  &  & H & E & L & L & O\\ 
\hspace{8.0 mm} 001 &  &  & H & E & L & L & O & \\
\hspace{8.0 mm} 010 &  & H & E & L & L & O &  & \\
\hspace{8.0 mm} 011 & H & E & L & L & O &  &  & \\
\hspace{8.0 mm} 100 & E & L & L & O &  &  &  & H\\
\hspace{8.0 mm} 101 & L & L & O &  &  &  & H & E\\
\hspace{8.0 mm} 110 & L & O &  &  &  & H & E & L\\
\hspace{8.0 mm} 111 & O &  &  &  & H & E & L & L\\
\end{tabular}
\end{center}

\begin{center}
Table 3. Rotating the word HELLO on eight displays.
\end{center}
~\\
~\\

Perform the following steps:
\begin{enumerate}
\item Create a new Quartus II project for your circuit and select as the target chip the 
Cyclone II EP2C35F672C6.
\item Include your VHDL entity in the Quartus II project. Connect the switches 
{\it SW}$_{17-15}$ to the select inputs of each instance of the 
multiplexers in your circuit. Also connect {\it SW}$_{14-0}$ to each instance of the
multiplexers as required to produce the patterns of characters shown in Table 3. (Hint: for some
inputs of the multiplexers you will want to select the `blank' character.)
Connect the outputs of your multiplexers to the 7-segment displays {\it HEX7}, $\ldots$, 
{\it HEX0}.
\item Include the required pin assignments for the DE2 board for all switches, LEDs, 
and 7-segment displays. Compile the project.
\item Download the compiled circuit into the FPGA chip. Test the functionality of the 
circuit by setting the proper character codes on the switches SW$_{14-0}$ and then 
toggling {\it SW}$_{17-15}$ to observe the rotation of the characters.
\end{enumerate}

~\\
~\\
~\\
~\\
~\\
~\\
~\\
~\\
~\\
~\\
~\\
Copyright \copyright 2006 Altera Corporation. 

\end{document}
