\documentclass[epsfig,10pt,fullpage]{article}
\addtolength{\textwidth}{1.5in}
\addtolength{\oddsidemargin}{-0.75in}
\addtolength{\topmargin}{-0.75in}
\addtolength{\textheight}{1.5in}
\addtolength{\evensidemargin}{0.75in}
\raggedbottom
\usepackage{ae,aecompl}
\usepackage{epsfig,float,times}

\begin{document}

\centerline{\huge Laboratory Exercise 3}
~\\
\centerline{\large Latches, Flip-flops, and Registers}
~\\
\noindent
The purpose of this exercise is to investigate latches, flip-flops, and registers.

~\\
\noindent
{\bf Part I}
~\\
~\\
\noindent
\noindent
Altera FPGAs include flip-flops that are available for implementing a user's circuit. 
We will show how to make use of these flip-flops in Parts IV to VII of this exercise. But
first we will show how storage elements can be created in an FPGA without using 
its dedicated flip-flops.

Figure 1 depicts a gated RS latch circuit. A style of VHDL code that 
uses logic expressions to describe this circuit is given in Figure 2.
If this latch is implemented
in an FPGA that has 4-input lookup tables (LUTs), then only one lookup table is needed, 
as shown in Figure 3$a$. 
~\\

\begin{figure}[H]
\scriptsize
\centerline{
\hbox{\psfig{figure=RS_latch.eps}}}
\end{figure}
~\\
\centerline{Figure 1.  A gated RS latch circuit.}
~\\
\begin{center}
\begin{minipage}[t]{12.5 cm}
\begin{tabbing}
-\,- A gated RS latch desribed the hard way\\
LIBRARY ieee;\\
USE ieee.std\_logic\_1164.all;\\
\\
ZZ\=PORT ~( \=Clk, R, S	\=: OUTZZ\=STD\_LOGIC;\kill
ENTITY part1 IS\\
\>PORT (\>Clk, R, S \>: IN	\>STD\_LOGIC;\\
\>\>Q \>: OUT \>STD\_LOGIC);\\
END part1;\\
\\
ARCHITECTURE Structural OF part1 IS\\
\>SIGNAL R\_g, S\_g, Qa, Qb : STD\_LOGIC ;\\
\>ATTRIBUTE keep : boolean;\\
\>ATTRIBUTE keep of R\_g, S\_g, Qa, Qb : SIGNAL IS true;\\
BEGIN	\\
\>R\_g $<$= R AND Clk;\\
\>S\_g $<$= S AND Clk;\\
\>Qa $<$= NOT (R\_g OR Qb);\\
\>Qb $<$= NOT (S\_g OR Qa);\\
~\\
\>Q $<$= Qa;\\
~\\
END Structural;\\

\end{tabbing}
\end{minipage}
\end{center}

\begin{center}
Figure 2. Specifying the RS latch by using logic expressions.
\end{center}

Although the latch can be correctly realized in one 4-input LUT, this implementation
does not allow its internal signals, such as $R\_g$ and $S\_g$, to be observed, because
they are not provided as outputs from the LUT. To preserve these internal signals in 
the implemented circuit, it is necessary to include a {\it compiler
directive} in the code. In Figure 2 the directive {\it keep} is
included by using a VHDL ATTRIBUTE statement; it instructs the Quartus II compiler to 
use separate logic elements for each of
the signals $R\_g, S\_g, $Q$a,$ and Q$b$. Compiling the code produces the circuit with four
4-LUTs depicted in Figure 3$b$. 
~\\

\begin{figure}[H]
\scriptsize
\centerline{
\hbox{\psfig{figure=figure3.3.eps}}}
\end{figure}
\centerline{Figure 3.  Implementation of the RS latch from Figure 1.}
~\\

Create a Quartus II project for the RS latch circuit as follows:

\begin{enumerate}
\item Create a new project for the RS latch. Select as the target chip the 
Cyclone II EP2C35F672C6, which is the FPGA chip on the Altera DE2 board. 
\item Generate a VHDL file with the code in Figure 2
and include it in the project. 
\item Compile the code. Use the Quartus II RTL Viewer tool to examine the gate-level
circuit produced from the code, and use the Technology Viewer tool 
to verify that the latch is implemented as shown in Figure 3$b$.
\item Create a Vector Waveform File (.vwf) which specifies the inputs and outputs
of the circuit. Draw waveforms for the $R$ and $S$ inputs and use the Quartus II Simulator to
produce the corresponding waveforms for {\it R\_g}, {\it S\_g}, Q{\it a}, and Q{\it b}.
Verify that the latch works as expected using both functional and timing simulation.
\end{enumerate}

\noindent
{\bf Part II}
~\\
~\\
\noindent
Figure 4 shows the circuit for a gated D latch.

~\\
\begin{figure}[H]
\scriptsize
\centerline{
\hbox{\psfig{figure=D_latch.eps}}}
\end{figure}
~\\
\centerline{Figure 4.  Circuit for a gated D latch.}
~\\

\noindent
Perform the following steps:

\begin{enumerate}
\item Create a new Quartus II project. Generate a VHDL file using the style of code 
in Figure 2 for the gated D latch. Use the {\it keep} directive to ensure
that separate logic elements are used to implement the signals $R, S\_g, R\_g, $Q$a,$ and
Q$b$.
\item Select as the target chip the Cyclone II EP2C35F672C6 and compile the code. Use the 
Technology Viewer tool to examine the implemented circuit.
\item Verify that the latch works properly for all input conditions by using functional 
simulation. Examine the timing characteristics of the circuit by using timing simulation.
\item Create a new Quartus II project which will be used for implementation of the gated D
latch on the DE2 board. This project should consist of a top-level entity that 
contains the appropriate input and output ports (pins) for the DE2 board. Instantiate your
latch in this top-level entity. Use switch {\it SW}$_0$ to drive the {\it D} input of the latch,
and use {\it SW}$_1$ as the {\it Clk} input. Connect the Q output to {\it LEDR}$_{0}$.
\item
Recompile your project and download the compiled circuit onto the DE2 board.
\item
Test the functionality of your circuit by toggling the $D$ and {\it Clk} switches and observing 
the Q output.
\end{enumerate}

\noindent
{\bf Part III}
~\\
~\\
\noindent
Figure 5 shows the circuit for a master-slave D flip-flop.
~\\

\begin{figure}[H]
\scriptsize
\centerline{
\hbox{\psfig{figure=figure3.5.eps}}}
\end{figure}
~\\
\centerline{Figure 5.  Circuit for a master-slave D flip-flop.}

~\\
\noindent
Perform the following:
\begin{enumerate}
\item Create a new Quartus II project. Generate a VHDL file that instantiates two
copies of your gated D latch entity from Part II to implement the master-slave flip-flop.
\item Include in your project the appropriate input and output ports for the Altera
DE2 board. Use switch {\it SW}$_0$ to drive the D input of the flip-flop,
and use {\it SW}$_1$ as the {\it Clock} input. Connect the Q output to {\it LEDR}$_{0}$.
\item
Compile your project.
\item Use the Technology Viewer to examine the D flip-flop circuit, and use
simulation to verify its correct operation.
\item
Download the circuit onto the DE2 board and test its functionality 
by toggling the $D$ and {\it Clock} switches and observing the Q output.
\end{enumerate}

~\\
\noindent
{\bf Part IV}
~\\
~\\
\noindent
Figure 6 shows a circuit with three different storage elements: a gated D latch, 
a positive-edge triggered D flip-flop, and a negative-edge triggered D flip-flop.
~\\

\begin{figure}[H]
\scriptsize
\centerline{
\hbox{\psfig{figure=figure3.6.eps}}}
\end{figure}
~\\
\centerline{Figure 6.  Circuit and waveforms for Part IV.}
~\\

\noindent
Implement and simulate this circuit using Quartus II software as follows:
\begin{enumerate}
\item Create a new project. 
\item Write a VHDL file that instantiates the three storage elements. For this part you
should no longer use the {\it keep} directive (that is, the VHDL ATTRIBUTE statement) 
from Parts I to III. Figure 7
gives a behavioral style of VHDL code that specifies the gated D latch in Figure 4.
This latch can be implemented in one 4-input lookup table. Use a similar style of
code to specify the flip-flops in Figure 6.
\item Compile your code and use the Technology Viewer to examine the implemented circuit.
Verify that the latch uses one lookup table and that the flip-flops are implemented using
the flip-flops provided in the target FPGA.
\item Create a Vector Waveform File (.vwf) which specifies the inputs and outputs
of the circuit. Draw the inputs $D$ and {\it Clock} as indicated in Figure 6.
Use functional simulation to obtain the three output signals.
Observe the different behavior of the three storage elements.
\end{enumerate}

\begin{center}
\begin{minipage}[t]{12.5 cm}
\begin{tabbing}
LIBRARY ieee ; \\
USE ieee.std\_logic\_1164.all ; \\
~\\
ENTITY latch IS \\
ZZ\=PORT ~( \=D,ClkZZ \=: OUTZZ\=STD\_LOGIC;\kill
\>PORT (	\>D, Clk \>: IN \>STD\_LOGIC ; \\
\>\>Q \>: OUT \>STD\_LOGIC) ; \\
END latch ;\\
~\\
ARCHITECTURE Behavior OF latch IS    \\
BEGIN\\
ZZ\=ZZ\=ZZ\=ZZ\=\kill
\>PROCESS ( D, Clk ) \\
\>BEGIN\\
\>\>IF Clk = '1' THEN \\
\>\>\>Q $<$= D ; \\
\>\>END IF ; \\
\>END PROCESS ; \\
END Behavior ;\\
\end{tabbing}
\end{minipage}
\end{center}
\begin{center}
Figure 7. A behavioral style of VHDL code that specifies a gated D latch.
\end{center}

~\\
\noindent
{\bf Part V}

~\\
\noindent
We wish to display the hexadecimal value of a 16-bit number $A$
on the four 7-segment displays, $HEX7-4$.
We also wish to display the
hex value of a 16-bit number $B$ on the four 7-segment displays,
$HEX3-0$. The values of $A$ and $B$ are inputs to the circuit which are
provided by means of switches $SW_{15-0}$. 
This is to be done by first setting the switches to the value of $A$ and
then setting the switches to the value of $B$; therefore, the value of $A$
must be stored in the circuit.
\begin{enumerate}
\item Create a new Quartus II project which will be used to implement the desired
circuit on the Altera DE2 board.
\item Write a VHDL file that provides the necessary functionality.  Use {\it KEY}$_0$ as
an active-low asynchronous reset, and use {\it KEY}$_1$ as a clock input.
Include the VHDL file in your project and compile the circuit.
\item Assign the pins on the FPGA to connect to the switches and 7-segment
displays, as indicated in the User Manual for the DE2 board.
\item Recompile the circuit and download it into the FPGA chip.
\item Test the functionality of your design by toggling the switches
and observing the output displays.
\end{enumerate}


~\\
~\\
Copyright \copyright 2006 Altera Corporation. 

\end{document}
