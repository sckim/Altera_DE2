\documentclass[psfig,10pt,fullpage]{article}
\addtolength{\textwidth}{1.5in}
\addtolength{\oddsidemargin}{-0.75in}
\addtolength{\topmargin}{-0.75in}
\addtolength{\textheight}{1.5in}
\addtolength{\evensidemargin}{0.75in}
\raggedbottom
\usepackage{psfig,float,times}

\begin{document}

~\\
\centerline{\huge Laboratory Exercise 4}
~\\
\centerline{\large Counters}
~\\
~\\

This is an exercise in using counters.

~\\
\noindent
{\bf Part I}

~\\
\noindent
Consider the circuit in Figure 1. It is a 4-bit synchronous counter which uses four
T-type flip-flops. The counter increments its count on each positive edge of the clock
if the Enable signal is asserted. The counter is reset to 0 by using the Reset signal.
You are to implement a 16-bit counter of this type.
~\\

\begin{figure}[H]
\scriptsize
\centerline{
\hbox{\psfig{figure=figure4.1.eps}}}
\end{figure}
\centerline{Figure 1.  A 4-bit counter.}
~\\

\begin{enumerate}
\item Write a VHDL file that defines a 16-bit counter by using the structure
depicted in Figure 1, and compile the circuit. How many logic elements (LEs) are 
used to implement your circuit? What is the maximum frequency, {\it Fmax}, at which
your circuit can be operated?
\item Simulate your circuit to verify its correctness.
\item Augment your VHDL file to use the pushbutton $KEY_0$ as the {\it Clock}
input, switches $SW_1$ and $SW_0$ as {\it Enable} and {\it Reset} inputs, and
7-segment displays {\it HEX3-0} to display the hexadecimal count as your circuit
operates. Make the necessary pin assignments and compile the circuit.
\item Implement your circuit on the DE2 board and test its functionality by operating
the implemented switches.
\item Implement a 4-bit version of your circuit and use the Quartus II RTL Viewer to
see how Quartus II software synthesized your circuit. What are the differences in
comparison with Figure 8?
\end{enumerate}

~\\
\noindent
{\bf Part II}

~\\
\noindent
Simplify your VHDL code so that the counter specification is based on
the VHDL statement
$$
{\rm Q <= Q + 1;}
$$
\noindent
Compile a 16-bit version of this counter and compare the number of LEs needed
and the {\it Fmax} that is attainable.
Use the RTL Viewer to see the structure of this implementation and comment on
the differences with the design from Part I.

\pagebreak
\noindent
{\bf Part III}

~\\
\noindent
Use an LPM from the Library of Parameterized modules to implement a 16-bit 
counter. Choose the LPM options to be consistent with the above design, i.e.
with enable and synchronous clear.
How does this version compare with the previous designs?

~\\
\noindent
{\bf Part IV}

~\\
\noindent
Design and implement a circuit that successively flashes digits 0 
through 9 on the 7-segment display $HEX0$. Each digit should be 
displayed for about one second. Use a counter to determine the one-second 
intervals. The counter should be incremented by the 50-MHz clock signal 
provided on the DE2 board. Do not derive any other clock signals in your design--make 
sure that all flip-flops in your circuit are clocked directly by the 50 MHz clock signal.

~\\
\noindent
{\bf Part V}

~\\
\noindent
Design and implement a circuit that displays the word HELLO, in
ticker tape fashion, on the eight 7-segment displays $HEX7-0$. 
Make the letters move from right to left in intervals of about one second.
The patterns that should be displayed in successive clock intervals are given in Table 1.

~\\
\begin{center}
\begin{tabular}{l|cccccccc}
Clock cycle & \multicolumn{8}{c}{Displayed pattern} \\
\hline
\hspace{8.0 mm} {\rule[0mm]{0mm}{5mm}0} &  &  &  & H & E & L & L & O\\ 
\hspace{8.0 mm} 1 &  &  & H & E & L & L & O & \\
\hspace{8.0 mm} 2 &  & H & E & L & L & O &  & \\
\hspace{8.0 mm} 3 & H & E & L & L & O &  &  & \\
\hspace{8.0 mm} 4 & E & L & L & O &  &  &  & H\\
\hspace{8.0 mm} 5 & L & L & O &  &  &  & H & E\\
\hspace{8.0 mm} 6 & L & O &  &  &  & H & E & L\\
\hspace{8.0 mm} 7 & O &  &  &  & H & E & L & L\\
\hspace{8.0 mm} 8 &  &  &  & H & E & L & L & O\\
\hspace{8.0 mm} $\ldots$ & \multicolumn{8}{l}{and so on}\\
\end{tabular}
\end{center}

\begin{center}
Table 1. Scrolling the word HELLO in ticker-tape fashion.
\end{center}
~\\
~\\
~\\
~\\
~\\
~\\
~\\
~\\
~\\
~\\
~\\
Copyright \copyright 2006 Altera Corporation. 

\end{document}
