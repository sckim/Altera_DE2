\documentclass[epsfig,10pt,fullpage]{article}
\addtolength{\textwidth}{1.5in}
\addtolength{\oddsidemargin}{-0.75in}
\addtolength{\topmargin}{-0.75in}
\addtolength{\textheight}{1.5in}
\addtolength{\evensidemargin}{0.75in}
\raggedbottom
\usepackage{ae,aecompl}
\usepackage{epsfig,float,times}

\begin{document}
~\\
\centerline{\huge Laboratory Exercise 10}
~\\
\centerline{\large An Enhanced Processor}
~\\
\noindent
In Laboratory Exercise 9 we described a simple processor. In
Part I of that exercise the processor itself was designed, and in Part II the processor was
connected to an external counter and a memory unit. This exercise describes subsequent
parts of the processor design. Note that the numbering of figures and tables in this exercise 
are continued from those in Parts I and II in the preceding lab exercise.
~\\
~\\
\noindent
{\bf Part III}
~\\
~\\
\noindent
In this part you will extend the capability of the processor so that the external counter 
is no longer needed, and so that the processor has the ability to perform read and
write operations using memory or other devices. You will add three new types of instructions
to the processor, as displayed in Table 3. The {\bf ld} (load) instruction loads data into register RX
from the external memory address specified in register RY. The {\bf st} (store) instruction
stores the data contained in register RX into the memory address found in RY. Finally, the
instruction {\bf mvnz} (move if not zero) allows a {\bf mv} operation to be executed only under a certain
condition; the condition is that the current contents of register $G$ are not equal to 0.

\begin{center}
\begin{tabular}{l|c}
\rule[-0.075in]{0in}{0.25in}Operation & Function performed \\ \hline 
\rule[-0.075in]{0in}{0.25in}ld $Rx$,$[Ry]$ & $Rx \leftarrow [[Ry]]$ \\ 
\rule[-0.075in]{0in}{0.25in}st $Rx$,$[Ry]$ & $[Ry] \leftarrow [Rx]$ \\ 
\rule[-0.075in]{0in}{0.25in}mvnz $Rx,Ry$ & if G != 0, $Rx \leftarrow [Ry]$ \\ 
\end{tabular}
\end{center}

\begin{center}
Table 3. New instructions performed in the processor.
\end{center}

A schematic of the enhanced processor is given in Figure~7. In this figure,
registers $R0$ to $R6$ are the same as in Figure~1 of Laboratory Exercise 9, but 
register $R7$ has been changed
to a counter. This counter is used to provide the addresses in the memory from which the
processor's instructions are read; in the preceding lab exercise, a counter 
external to the processor was
used for this purpose. We will refer to $R7$ as the processor's 
{\it program counter} ({\it PC}), because this terminology is common for real processors
available in the industry. When the processor is reset, {\it PC} is set to address 0. At the
start of each instruction (in time step 0) the contents of {\it PC} are used as an address to
read an instruction from the memory. The instruction is stored in IR and 
the {\it PC} is automatically incremented to point to the next instruction (in the case of {\bf
mvi}
the {\it PC} provides the address of the immediate data and is then incremented again). 

The processor's control unit increments {\it PC} by using the {\it incr\_PC} signal, which is
just an enable on this counter. It is also
possible to directly load an address into {\it PC} ($R7$) by having the processor execute a {\bf
mv} or {\bf mvi}
instruction in which the destination register is specified as $R7$. In this case the control
unit uses the signal {\it R7}$_{in}$ to perform a parallel load of the counter. In this way, the
processor can execute instructions at any address in memory, as opposed to only being able
to execute instructions that are stored in successive addresses. Similarly, the current 
contents of {\it PC} can be copied into another register by using a {\bf mv} instruction. An example
of code that uses the {\it PC} register to implement a loop is shown below, where the 
text after the \% on each line is just a comment. The instruction
{\bf mv} R5,R7 places into {\it R5} the address in memory of the instruction {\bf sub} R4,R2.
Then, the instruction 
{\bf mvnz} R7,R5 causes the {\bf sub} instruction to be executed repeatedly until R4 becomes 0.
This type of loop could be used in a larger program as a way of creating a delay. 
~\\
\begin{center}
\begin{minipage}[t]{12.5 cm}
\begin{tabbing}
ZZZZZ\=ZZZZZZZZZZZ\=ZZ\=ZZ\=ZZ\=ZZ\=ZZ\=ZZ\=ZZ\kill
{\bf mvi} \>R2,\#1 \\
{\bf mvi} \>R4,\#10000000 \>\% binary delay value\\
{\bf mv} \>R5,R7 \>\% save address of next instruction\\
{\bf sub} \>R4,R2 \>\% decrement delay count\\
{\bf mvnz} \>R7,R5 \>\% continue subtracting until delay count gets to 0\\
\end{tabbing}
\end{minipage}
\end{center}

\begin{figure}[H]
\scriptsize
\centerline{
%\hbox{\psfig{figure=figure7.eps,width=6.5in}}}
\hbox{\psfig{figure=figure7.eps}}}
\end{figure}
\centerline{Figure 7. An enhanced version of the processor.}
~\\

Figure 7 shows two registers in the processor that are used for data transfers. The 
{\it ADDR}
register is used to send addresses to an external device, such as a memory module, and the
{\it DOUT} register is used by the processor to provide data that can be stored outside the
processor. One use of the {\it ADDR} register is for reading, or {\it fetching}, instructions 
from memory; when the processor wants to fetch an instruction, the contents of {\it PC} ($R7$) are
transferred across the bus and loaded into {\it ADDR}. This address is provided to memory. 
In addition to
fetching instructions, the processor can read data at any address by using the {\it ADDR}
register. Both data and instructions are read into the processor on the {\it DIN} input port.
The processor can write data for storage at an external address by placing this address 
into the {\it ADDR} register, placing the data to be stored into its {\it DOUT} register, 
and asserting the output of the {\it W} (write) flip-flop to 1. 

Figure 8 illustrates how the enhanced processor is connected to memory and other devices.
The memory unit in the figure supports both read and write operations and therefore has 
both address and data inputs, as well as a write enable input. The memory also has a clock
input, because the address, data, and write enable inputs must be loaded into the memory
on an active clock edge. This type of memory unit is usually called a {\it synchronous random
access memory (synchronous RAM)}. Figure~8 also includes a 16-bit register that can be used
to store data from the processor; this register might be connected to a set of
LEDs to allow display of data on the DE2 board. To allow the processor to select either the
memory unit or register when performing a write operation, the circuit includes some logic gates
that perform {\it address decoding}: if the upper address lines are
$A_{15} A_{14} A_{13} A_{12} = 0000$, then the memory module will be written at the
address given on the lower address lines. Figure~8 shows $n$ lower address lines connected
to the memory; for this exercise a memory with 128 words is probably sufficient, which
implies that $n = 7$ and the memory address port is driven by $A_6 \ldots A_0$. For
addresses in which $A_{15} A_{14} A_{13} A_{12} = 0001$, the data written by the processor
is loaded into the register whose outputs are called {\it LEDs} in Figure~8.
~\\
\begin{figure}[H]
\scriptsize
\centerline{
\hbox{\psfig{figure=figure8.eps}}}
\end{figure}
\centerline{Figure 8. Connecting the enhanced processor to a memory and output register.}

\begin{enumerate}
\item Create a new Quartus II project for the enhanced version of the processor.
\item Write VHDL code for the processor and test your circuit by using functional
simulation: apply instructions to the {\it DIN} port and observe the internal processor
signals as the instructions are executed. Pay careful attention to the timing of 
signals between your processor and external memory; account for the fact that the 
memory has registered input ports, as we discussed for Figure~8. 
\item Create another Quartus II project that instantiates the processor, memory module, and
register shown in Figure~8. Use the Quartus II MegaWizard Plug-In Manager tool to 
create the ALTSYNCRAM memory module. Follow the
instructions provided by the wizard to create a memory that has one 16-bit wide read/write
data port and is 128 words deep. Use a MIF file to store instructions in the 
memory that are to be executed by your processor.
\item Use functional simulation to test the circuit. Ensure that data is read properly
from the RAM and executed by the processor.
\item Include in your project the necessary pin assignments to implement your circuit on the DE2
board. Use switch {\it SW}$_{17}$ to drive the processor's {\it Run} input, use 
{\it KEY}$_0$ for {\it Resetn}, and use the board's 50 MHz clock signal 
as the {\it Clock} input. Since the circuit needs to run properly at 50 MHz, make sure that
a timing constraint is set in Quartus II to constrain the circuit's clock to this
frequency. Read the Report produced by the Quartus II Timing Analyzer to ensure that
your circuit operates at this speed; if not, use the Quartus II tools to analyze your
circuit and modify your VHDL code to make a more efficient design that meets the 50-MHz
speed requirement. Also note that the {\it Run} input is asynchronous to the clock signal,
so make sure to synchronize this input using flip-flops.

Connect the {\it LEDs} register in Figure~8 to {\it LEDR}$_{15-0}$ so that you can 
observe the output produced by the processor.
\item Compile the circuit and download it into the FPGA chip.
\item Test the functionality of your design by executing code from the RAM
and observing the LEDs. 
\end{enumerate}
~\\
\noindent
{\bf Part IV}
~\\
~\\
\noindent
In this part you are to connect an additional I/O module to your circuit from Part III
and write code that is executed by your processor.

Add a module called {\it seg7\_scroll} to your circuit. This module should contain one
register for each 7-segment display on the DE2 board. Each register should directly drive
the segment lights for one 7-segment display, so that the processor can write characters
onto these displays. Create the necessary address decoding to allow the processor to
write to the registers in the {\it seg7\_scroll} module.

\begin{enumerate}
\item Create a Quartus II project for your circuit and write the VHDL code that
includes the circuit from Figure~8 in addition to your {\it seg7\_scroll} module.
\item Use functional simulation to test the circuit.
\item Add appropriate timing constraints and pin assignments to 
your project, and write a MIF file that allows the processor to write characters to the 
7-segment displays. A simple program would write a word to the displays and then
terminate, but a more interesting program could scroll a message across the displays, or 
scroll a word across the displays in the left, right, or both directions.
\item Test the functionality of your design by executing code from the RAM
and observing the 7-segment displays.
\end{enumerate}

\noindent
{\bf Part V}
~\\
~\\
\noindent
Add to your circuit from Part IV another module, called {\it port\_n}, that allows the 
processor to read the state of some switches on the board. The switch values should be 
stored into a register, and the processor should be able to read this register by 
using a {\bf ld} instruction. You will have to
use address decoding and multiplexers to allow the processor to read from either the RAM
or {\it port\_n} units, according to the address used.

\begin{enumerate}
\item Draw a circuit diagram that shows how the {\it port\_n} unit is incorporated into the system.
\item Create a Quartus II project for your circuit, write the VHDL code, and write a MIF
file that demonstrates use of the {\it port\_n} module.
One interesting application is to have the processor scroll a message across 
the 7-segment displays and use
the values read from the {\it port}\_{\it n} module to change the speed at which the message
is scrolled.
\item Test your circuit both by using functional simulation and by downloading it and executing
your processor code on the DE2 board.
\end{enumerate}
\noindent
{\bf Suggested Bonus Parts}
~\\
~\\
\noindent
The following are suggested bonus parts for this exercise.
\begin{enumerate}
\item Use the Quartus II tools to identify the critical paths in the processor circuit.
Modify the processor design so that the circuit will operate at the highest clock
frequency that you can achieve.
\item Extend the instructions supported by your processor to make it more flexible. Some suggested
instruction types are logic instructions (AND, OR, etc), shift instructions, and branch
instructions. You may also wish to add support for logical conditions other than ``not zero''
, as supported by {\bf mvnz}, and the like.
\item Write an Assembler program for your processor. It should automatically produces a MIF file
from assembler code.
\end{enumerate}
Copyright \copyright 2006 Altera Corporation
\end{document}
