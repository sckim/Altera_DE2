\documentclass[epsfig,10pt,fullpage]{article}
\addtolength{\textwidth}{1.5in}
\addtolength{\oddsidemargin}{-0.75in}
\addtolength{\topmargin}{-0.75in}
\addtolength{\textheight}{1.5in}
\addtolength{\evensidemargin}{0.75in}
\raggedbottom
\usepackage{ae,aecompl}
\usepackage{epsfig,float,times}

\begin{document}
~\\
\centerline{\huge Laboratory Exercise 9}
~\\
\centerline{\large A Simple Processor}
~\\
~\\
\noindent
Figure~1 shows a digital system that contains a number of 16-bit registers, a multiplexer, an
adder/subtracter unit, a counter, and a control unit. Data is input to this system via the 16-bit
{\it DIN} input. This data can be loaded through the 16-bit wide multiplexer into the various
registers, such as $R0, \ldots, R7$ and {\it A}. The multiplexer also allows 
data to be transferred from one register to another. The multiplexer's output wires are called 
a {\it bus} in the figure because this term is often used for wiring that allows data to be 
transferred from one location in a system to another.

Addition or subtraction is performed by using the multiplexer to first place one
16-bit number onto the bus wires and loading this number into register {\it A}. Once this
is done, a second 16-bit number is placed onto the bus, the adder/subtracter unit
performs the required operation, and the result is loaded into register {\it G}. The
data in {\it G} can then be transferred to one of the other registers as required.

\begin{figure}[H]
\scriptsize
\centerline{
%\hbox{\psfig{figure=figure1.eps,width=6.25in}}}
\hbox{\psfig{figure=figure1.eps}}}
\end{figure}
\centerline{Figure 1.  A digital system.}
~\\

The system can perform different operations in each clock cycle, 
as governed by the {\it control unit}. This unit determines when particular data is 
placed onto the bus wires and it controls which of the registers is to be loaded with this data.
For example, if the control unit asserts the signals 
$R0_{out}$ and $A_{in}$, then the 
multiplexer will place the contents of register $R0$ onto the bus 
and this data will be loaded by the next active clock edge into register {\it A}.

A system like this is often called a {\it processor}. It executes operations specified in
the form of instructions. 
Table~1 lists the instructions that the processor has to support for this exercise.
The left column shows the name of an instruction and its operand. The meaning of the syntax 
RX $\leftarrow$ [RY] is that the contents of register RY are loaded into register RX. 
The {\bf mv} (move) instruction allows data to be copied from one register to another. 
For the {\bf mvi} (move immediate) instruction the expression RX $\leftarrow$ D indicates that the
16-bit constant D is loaded into register RX.

\begin{center}
\begin{tabular}{l|c}
\rule[-0.075in]{0in}{0.25in}Operation & Function performed \\ \hline 
\rule[-0.075in]{0in}{0.25in}{\bf mv} $Rx$,$Ry$ & $Rx \leftarrow [Ry]$ \\ 
\rule[-0.075in]{0in}{0.25in}{\bf mvi} $Rx$,\#$D$ & $Rx \leftarrow D$ \\ 
\rule[-0.075in]{0in}{0.25in}{\bf add} $Rx,Ry$ & $Rx \leftarrow [Rx] + [Ry]$ \\ 
\rule[-0.075in]{0in}{0.25in}{\bf sub} $Rx,Ry$ & $Rx \leftarrow [Rx] - [Ry]$ \\ 
\end{tabular}
\end{center}

\begin{center}
Table 1. Instructions performed in the processor.
\end{center}

Each instruction can be encoded and stored in the {\it IR} register using the 9-bit format 
IIIXXXYYY, where III represents the instruction, XXX gives the RX register, and YYY
gives the RY register. Although only two bits are needed to encode our four
instructions, we are using three bits because other instructions will be added to the
processor in later parts of this exercise. Hence {\it IR} has to be connected to nine bits of the 
16-bit {\it DIN} input, as indicated in Figure~1.
For the {\bf mvi} instruction the YYY field has no meaning, and the immediate data \#D has to be
supplied on the 16-bit {\it DIN} input after the {\bf mvi} instruction word is stored into {\it IR}.

Some instructions, such as an addition or subtraction, take more than one clock cycle to
complete, because multiple transfers have to be performed across the bus. The control unit uses 
the two-bit counter shown in Figure~1 to enable it to ``step through'' such instructions.
The processor starts executing the instruction on the {\it DIN} input when the {\it Run}
signal is asserted 
and the processor asserts the {\it Done} output when the instruction is finished.
Table 2 indicates the control signals that can be asserted in each time step to implement
the instructions in Table 1. Note that the only control signal asserted in time step 0 is
{\it IR}$_{in}$, so this time step is not shown in the table.

\begin{center}
\begin{tabular}{c|c|c|c|}
 \multicolumn{1}{c}{~} & \multicolumn{1}{c}{$T_1$} & \multicolumn{1}{c}{$T_2$} & \multicolumn{1}{c}{$T_3$} \rule[-0.075in]{0in}{0.25in}\\ \cline{2-4}
({\bf mv}): $I_0$ & \rule[-0.075in]{0in}{0.25in}{\it RY$_{out}$}, {\it RX$_{in}$}, &  &  \\ 
~ & {\it Done} &  &  \\ \cline{2-4}
\multicolumn{1}{c|}{({\bf mvi}): $I_1$} & \rule[-0.075in]{0in}{0.25in}{\it DIN$_{out}$},
{\it RX$_{in}$}, &  &  \\ 
~ & {\it Done} &  &  \\ \cline{2-4}
\rule[-0.075in]{0in}{0.25in}({\bf add}): $I_2$ & {\it RX$_{out}$}, {\it A$_{in}$} & {\it
RY$_{out}$}, {\it G$_{in}$} & {\it G$_{out}$}, {\it RX$_{in}$}, \\
~ & ~ & ~ & {\it Done} \\
\cline{2-4}
\rule[-0.075in]{0in}{0.25in}({\bf sub}): $I_3$ & {\it RX$_{out}$}, {\it A$_{in}$} & {\it
RY$_{out}$}, {\it
G$_{in}$}, & {\it G$_{out}$}, {\it RX$_{in}$}, \\
~ & ~ & {\it AddSub} & {\it Done} \\
\cline{2-4}
\end{tabular}
\end{center}

\begin{center}
Table 2. Control signals asserted in each instruction/time step.
\end{center}

\pagebreak
\noindent
{\bf Part I}
~\\
~\\
\noindent
Design and implement the processor shown in Figure~1 using VHDL code as follows:
\begin{enumerate}
\item Create a new Quartus II project for this exercise.
\item Generate the required VHDL file, include it in your project, 
and compile the circuit. A suggested skeleton of the VHDL code is shown in parts $a$ and
$b$ of Figure~2,
and some subcircuit entities that can be used in this code appear in parts $c$ and $d$.
\item Use functional simulation to verify that your code is correct. An example of the output
produced by a functional simulation for a correctly-designed circuit is given in Figure~3.
It shows the value ${(2000)}_{16}$ being loaded into {\it IR} from {\it DIN} at time 30 ns. This
pattern represents the instruction {\bf mvi} R0,\#D, where the value $D = 5$ is loaded
into $R0$ on
the clock edge at 50 ns. The simulation then shows the instruction {\bf mv} R1,R0 at 90
ns, {\bf add} R0,R1 at 110 ns, and {\bf sub} R0,R0 at 190 ns. 
Note that the simulation output shows {\it DIN} as a 4-digit
hexadecimal number, and it shows the contents of {\it IR} as a 3-digit octal number.
\item Create a new Quartus II project which will be used for implementation of the 
circuit on the Altera DE2 board. This project should consist of a top-level entity that
contains the appropriate input and output ports for the Altera board. Instantiate your
processor in this top-level entity. Use switches {\it SW}$_{15-0}$ to drive the {\it DIN} input 
port of the processor and use switch {\it SW}$_{17}$ to drive the {\it Run} input. Also, use 
push button {\it KEY}$_0$ for {\it Resetn} and {\it KEY}$_1$ for {\it Clock}.
Connect the processor bus wires to {\it LEDR}$_{15-0}$ and connect the {\it Done} signal to
{\it LEDR}$_{17}$.
\item Add to your project the necessary pin assignments for the 
DE2 board.  Compile the circuit and download it into the FPGA chip.
\item Test the functionality of your design by toggling the switches
and observing the LEDs. Since the processor's clock input is controlled by a push button
switch, it is easy to step through the execution of instructions and observe the behavior
of the circuit.
\end{enumerate}

\begin{center}
\begin{minipage}[t]{12.5 cm}
\begin{tabbing}
LIBRARY ieee;
USE ieee.std\_logic\_1164.all;\\
USE ieee.std\_logic\_signed.all;\\
~\\
ENTITY proc IS\\
ZZ\=PORT ~( \=Reset, Clock, RunZZ \=: BUFFERZZ\=STD\_LOGIC;\kill
\>PORT ( \>DIN \>: IN \>STD\_LOGIC\_VECTOR(15 DOWNTO 0);\\
\>\>Resetn, Clock, Run \>: IN \>STD\_LOGIC;\\
\>\>Done \>: BUFFER \>STD\_LOGIC;\\
\>\>BusWires \>: BUFFER	\>STD\_LOGIC\_VECTOR(15 DOWNTO 0));\\
END proc;\\
~\rule{5.0in}{0in}
~\\
ARCHITECTURE Behavior OF proc IS\\
ZZ\=ZZ\=ZZ\=ZZ\=ZZ\=ZZ\=\kill
\>$\ldots$ declare components\\
\>$\ldots$ declare signals\\
BEGIN\\
\>High $<$= '1';\\
\>Clear $<$= $\ldots$\\\\
\>Tstep: upcount PORT MAP (Clear, Clock, Tstep\_Q);\\
\>I $<$= IR(1 TO 3);\\
\>decX: dec3to8 PORT MAP (IR(4 TO 6), High, Xreg);\\
\>decY: dec3to8 PORT MAP (IR(7 TO 9), High, Yreg);\\
~\rule{5.0in}{0in}\\
\end{tabbing}
\end{minipage}
\end{center}

\begin{center}
Figure 2$a$. Skeleton VHDL code for the processor.
\end{center}

\begin{center}
\begin{minipage}[t]{12.5 cm}
\begin{tabbing}
ZZ\=ZZ\=ZZ\=ZZ\=ZZ\=ZZ\=\kill
\>controlsignals: PROCESS (Tstep\_Q, I, Xreg, Yreg)\\
\>BEGIN\\
\>\>$\ldots$ specify initial values\\
\>\>CASE Tstep\_Q IS\\
\>\>\>WHEN "00" =$>$ -\,- store DIN in IR as long as Tstep\_Q = 0\\
\>\>\>\>IRin $<$= '1';\\
\>\>\>WHEN "01" =$>$ -\,- define signals in time step T1\\
\>\>\>\>CASE I IS\\
\>\>\>\>\>$\ldots$ \\
\>\>\>\>END CASE;\\
\>\>\>WHEN "10" =$>$ -\,- define signals in time step T2\\
\>\>\>\>CASE I IS\\
\>\>\>\>\>$\ldots$ \\
\>\>\>\>END CASE;\\
\>\>\>WHEN "11" =$>$ -\,- define signals in time step T3\\
\>\>\>\>CASE I IS\\
\>\>\>\>\>$\ldots$ \\
\>\>\>\>END CASE;\\
\>\>END CASE;\\
\>END PROCESS;	\\
~\\
\>reg\_0: regn PORT MAP (BusWires, Rin(0), Clock, R0);\\
\>$\ldots$ instantiate other registers and the adder/subtracter unit\\
\>$\ldots$ define the bus\\
END Behavior;\\
~\rule{5.0in}{0in}
\end{tabbing}
\end{minipage}
\end{center}
\begin{center}
Figure 2$b$. Skeleton VHDL code for the processor.
\end{center}

\begin{center}
\begin{minipage}[t]{12.5 cm}
\begin{tabbing}
LIBRARY ieee;\\
USE ieee.std\_logic\_1164.all;\\
USE ieee.std\_logic\_signed.all;\\
~\\
ZZ\=PORT ~( \=Clock, ClockZZ \=: OUTZZ\=STD\_LOGIC;\kill
ENTITY upcount IS\\
\>PORT ( \>Clear, Clock \>: IN \>STD\_LOGIC;\\
\>\>Q \>: OUT \>STD\_LOGIC\_VECTOR(1 DOWNTO 0));\\
END upcount;\\
~\\
ZZ\=ZZ\=ZZ\=ZZ\=ZZ\=ZZ\=\kill
ARCHITECTURE Behavior OF upcount IS\\
\>SIGNAL Count : STD\_LOGIC\_VECTOR(1 DOWNTO 0);\\
BEGIN\\
\>PROCESS (Clock)\\
\>BEGIN\\
\>\>IF (Clock'EVENT AND Clock = '1') THEN\\
\>\>\>IF Clear  = '1' THEN\\
\>\>\>\>Count $<$= "00";\\
\>\>\>ELSE \\
\>\>\>\>Count $<$= Count + 1;\\
\>\>\>END IF;\\
\>\>END IF;\\
\>END PROCESS;\\
\>Q $<$= Count;\\
END Behavior;\\
~\rule{5.0in}{0in}
\end{tabbing}
\end{minipage}
\end{center}

\begin{center}
Figure 2$c$. Subcircuit entities for use in the processor.
\end{center}

\begin{center}
\begin{minipage}[t]{12.5 cm}
\begin{tabbing}
~\\
LIBRARY ieee;\\
USE ieee.std\_logic\_1164.all;\\
~\\
ZZ\=PORT ~( \=WZZ \=: OUTZZ\=STD\_LOGIC;\kill
ENTITY dec3to8 IS\\
\>PORT ( \>W \>: IN \>STD\_LOGIC\_VECTOR(2 DOWNTO 0);\\
\>\>En \>: IN \>STD\_LOGIC;\\
\>\>Y \>: OUT \>STD\_LOGIC\_VECTOR(0 TO 7));\\
END dec3to8;\\
~\\
ZZ\=ZZ\=ZZ\=ZZ\=ZZ\=ZZ\=\kill
ARCHITECTURE Behavior OF dec3to8 IS\\
BEGIN\\
\>PROCESS (W, En)\\
\>BEGIN\\
\>\>IF En = '1' THEN\\
\>\>\>CASE W IS\\
\>\>\>\>WHEN "000" =$>$ Y $<$= "10000000";\\
\>\>\>\>WHEN "001" =$>$ Y $<$= "01000000";\\
\>\>\>\>WHEN "010" =$>$ Y $<$= "00100000";\\
\>\>\>\>WHEN "011" =$>$ Y $<$= "00010000";\\
\>\>\>\>WHEN "100" =$>$ Y $<$= "00001000";\\
\>\>\>\>WHEN "101" =$>$ Y $<$= "00000100";\\
\>\>\>\>WHEN "110" =$>$ Y $<$= "00000010";\\
\>\>\>\>WHEN "111" =$>$ Y $<$= "00000001";\\
\>\>\>END CASE;\\
\>\>ELSE \\
\>\>\>Y $<$= "00000000";\\
\>\>END IF;\\
\>END PROCESS;\\
END Behavior;\\
~\\
LIBRARY ieee;\\
USE ieee.std\_logic\_1164.all;\\
~\\
ZZ\=PORT ~( \=Rin, ClockZZ \=: BUFFERZZ\=STD\_LOGIC;\kill
ENTITY regn IS\\
\>GENERIC (n : INTEGER := 16);\\
\>PORT ( \>R \>: IN \>STD\_LOGIC\_VECTOR(n-1 DOWNTO 0);\\
\>\>Rin, Clock \>: IN \>STD\_LOGIC;\\
\>\>Q \>: BUFFER \>STD\_LOGIC\_VECTOR(n-1 DOWNTO 0));\\
END regn;\\
~\\
ZZ\=ZZ\=ZZ\=ZZ\=ZZ\=ZZ\=\kill
ARCHITECTURE Behavior OF regn IS\\
BEGIN\\
\>PROCESS (Clock)\\
\>BEGIN\\
\>\>IF Clock'EVENT AND Clock = '1' THEN\\
\>\>\>IF Rin = '1' THEN\\
\>\>\>\>Q $<$= R;\\
\>\>\>END IF;\\
\>\>END IF;\\
\>END PROCESS;\\
END Behavior;\\
~\rule{5.0in}{0in}
\end{tabbing}
\end{minipage}
\end{center}

\begin{center}
Figure 2$d$. Subcircuit entities for use in the processor.
\end{center}

~\\
\begin{figure}[H]
\scriptsize
\centerline{
\hbox{\psfig{figure=figure3.eps}}}
\end{figure}
\centerline{Figure 3.  Simulation of the processor.}

~\\
\noindent
{\bf Part II}
~\\
~\\
\noindent
In this part you are to design the circuit depicted in Figure~4, in which a 
memory module and counter are connected to the processor from Part I. The
counter is used to read the contents of successive addresses in the memory, and
this data is provided to the processor as a stream of instructions. To simplify the
design and testing of this circuit we have used separate clock signals, {\it PClock} 
and {\it MClock}, for the processor and memory.
\begin{figure}[H]
\scriptsize
\centerline{
\hbox{\psfig{figure=figure4.eps}}}
\end{figure}
\centerline{Figure 4. Connecting the processor to a memory and counter.}
\begin{enumerate}
\item Create a new Quartus II project which will be used to test your circuit.
\item Generate a top-level VHDL file that instantiates the processor, memory, and
counter. Use the Quartus II MegaWizard Plug-In Manager tool to create the memory 
module from the Altera library of parameterized modules (LPMs). The correct
LPM is found under the {\it storage} category and is called {\it ALTSYNCRAM}. Follow the
instructions provided by the wizard to create a memory that has one 16-bit wide read data port
and is 32 words deep. The first screen of the wizard is shown in Figure~5. Since
this memory has only a read port, and no write port, it is called a 
{\it synchronous read-only memory (synchronous ROM)}. 
Note that the memory includes a register for synchronously loading addresses. This 
register is required
due to the design of the memory resources on the Cyclone II FPGA; account for
the clocking of this address register in your design.

To place processor instructions into the memory, you
need to specify {\it initial values} that should be stored in the memory once your circuit
has been programmed into the FPGA chip. This can be done by telling the wizard
to initialize the memory using the contents of a {\it memory initialization file (MIF)}. The 
appropriate screen of the MegaWizard Plug-In Manager tool is illustrated in
Figure~6. We have specified a file named {\it inst\_mem.mif}, which then 
has to be created in the directory that contains the Quartus II project. Use the Quartus II 
on-line Help to learn about the format of the {\it MIF} file and create a file that has enough
processor instructions to test your circuit.
\item Use functional simulation to test the circuit. Ensure that data is read properly
out of the ROM and executed by the processor.
\item 
Make sure your project includes the necessary port names and pin location assignments 
to implement the circuit on the DE2
board. Use switch {\it SW}$_{17}$ to drive the processor's {\it Run} input, use 
{\it KEY}$_0$ for {\it Resetn}, use {\it KEY}$_1$ for {\it MClock},
and use {\it KEY}$_2$ for {\it PClock}.
Connect the processor bus wires to {\it LEDR}$_{15-0}$ and connect the {\it Done}
signal to {\it LEDR}$_{17}$.
\item Compile the circuit and download it into the FPGA chip.
\item Test the functionality of your design by toggling the switches
and observing the LEDs. Since the circuit's clock inputs are controlled by push button
switches, it is easy to step through the execution of instructions and observe the behavior
of the circuit.
\end{enumerate}
~\\
\begin{figure}[H]
\scriptsize
\centerline{
\hbox{\psfig{figure=figure5.eps}}}
\end{figure}
\centerline{Figure 5. ALTSYNCRAM configuration.}
\begin{figure}[H]
\scriptsize
\centerline{
\hbox{\psfig{figure=figure6.eps}}}
\end{figure}
\centerline{Figure 6. Specifying a memory initialization file (MIF).}

~\\
\noindent
{\bf Enhanced Processor}
~\\
~\\
\noindent
It is possible to enhance the capability of the processor so that the counter in 
Figure~4 is no longer needed, and so that the processor has the ability to perform read and
write operations using memory or other devices. These enhancements involve adding new
instructions to the processor and the programs that the processor executes are therefore
more complex. Since these steps are beyond the scope of some logic design courses, they
are described in a subsequent lab exercise available from Altera.

~\\
~\\
~\\
~\\
~\\
Copyright \copyright 2006 Altera Corporation.
\end{document}
